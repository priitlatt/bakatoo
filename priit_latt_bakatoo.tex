\documentclass[a4paper,12pt]{article}

\usepackage{tensor}
\usepackage{amsmath}
\usepackage{amsthm}
\usepackage{amsfonts}
\usepackage{amssymb}
\usepackage[utf8]{inputenc}
\usepackage[estonian]{babel}
\usepackage[a4paper]{geometry}
\usepackage{nameref}
\usepackage[stable]{footmisc}
\usepackage{hyperref}
\usepackage{verbatim}
\usepackage[titletoc, toc, page]{appendix}
\usepackage{color}

\renewcommand{\appendixtocname}{Lisad}
\renewcommand{\appendixname}{Lisa}

\addto\captionsestonian{%
  \renewcommand\appendixname{Lisa}
  \renewcommand\appendixpagename{Lisad}
}

\theoremstyle{plain}
\newtheorem{teoreem}{Teoreem}[section]
\newtheorem{lemma}{Lemma}[section]
\newtheorem{lause}{Lause}[section]
\newtheorem{algoritm}{Algoritm}[section]
\newtheorem{jareldus}{Järeldus}[section]
\newtheorem{hupotees}{Hüpotees}[section]

\theoremstyle{definition}
\newtheorem{definitsioon}{Definitsioon}[section]
\newtheorem{naide}{Näide}[section]
\newtheorem{markus}{Märkus}[section]
\newtheorem{tahistus}{Tähistus}[section]
\newtheorem{kommentaar}{Kommentaar}[section]

\numberwithin{equation}{section}

\def\N{{\mathbb N}}
\def\R{{\mathbb R}}
\def\C{{\mathbb C}}
\def\K{{\mathbb K}}
\def\Z{{\mathbb Z}}
\def\Q{{\mathbb Q}}
\def\L{{\mathcal L}}
\def\M{{\mathcal M}}
\def\P{{\mathcal P}}
\def\su2{{\mathfrak{ su}\left(2\right)}}

\usepackage{mathtools}
\DeclareMathOperator{\spn}{span}
\DeclareMathOperator{\sign}{sign}
\DeclareMathOperator{\Mat}{Mat}
\DeclareMathOperator{\Tr}{Tr}
\DeclareMathOperator{\ad}{ad}
\DeclareMathOperator{\Lin}{Lin}

\title{Minkowski aegruumi geomeetriast}
%\author[P. Lätt]{Priit Lätt}
%\email[P. Lätt]{lattpriit@gmail.com}

\pagestyle{plain}
%\pagestyle{headings}

\begin{document}

\begin{titlepage}
\begin{center}

{\large TARTU ÜLIKOOL}\\[0.3cm]
{\large MATEMAATIKA-INFORMAATIKATEADUSKOND}\\[0.3cm]
{\large Matemaatika instituut}\\[0.3cm]
{\large Matemaatika eriala}\\[3cm]

{\large Priit Lätt}\\[0.3cm]
{\huge \textbf{Minkowski aegruumi geomeetriast}}\\[0.3cm]
{\large Bakalaureusetöö (6 EAP)}\\[3cm]

\begin{flushright}
{\large Juhendaja: Viktor Abramov}
\end{flushright}

\vfill

\begin{flushleft}
{\large
Autor: \dotfill "....." juuni 2013\\
Juhendaja: \dotfill "....." juuni 2013\\[2cm]
}
\end{flushleft}
{\large TARTU \the\year}

\end{center}
\end{titlepage}

\tableofcontents
\newpage

\section{Sissejuhatus}

Bakalaureusetöö on referatiivne ja selle aluseks on~\cite{Naber}. \\
Töös kasutame summade tähistamisel sageli Einstein'i summeerimiskokkulepet. See tähendab, kui meil on indeksid $i$ ja $j$, mis omavad väärtusi $1, \dots, n \left( n \in \N \right)$, siis kirjutame 
\begin{align*}
x^{i} e_{a} &= \sum_{a=1}^{n} x^{i} e_{i} = x^{1} e_{1} + x^{2} e_{2} + \dots + x^{n} e_{n}, \\
\lambda\indices{^i_j} x^{j} &= \sum_{j=1}^{n} \lambda\indices{^i_j} x^{j} = \lambda\indices{^i_1} x^{1} + \lambda\indices{^i_2} x^{2} + \dots +\lambda\indices{^i_n} x^{n},\\
\eta_{ij} u^{i} v^{j} &= \eta_{11} u^{1} v^{1} + \eta_{12} u^{1} v^{2} + \dots + \eta_{1n} u^{1} v^{n} + \eta_{21} u^{2} v^{1} + \dots + \eta_{nn} u^{n} v^{n},
\end{align*}
ja nii edasi.\\
Vektori $u$ pikkust tähistame edaspidi $|u|$. \\
Seda teksti pole tegelikutl vaja.\footnote{Hermann Minkowski (1864 - 1909) - poola matemaatik}


\newpage
\section{Vajalikud eelteadmised}

Selles peatükis toome välja definitsioonid ja tulemused, mida läheb tarvis töö järgmistes osades. Lihtsamad tulemused, millele on pööratud tähelepanu kursustes Algebra I või Geomeetria II, esitame seejuures tõestusteta.

\subsection{Skalaarkorrutisega seotud abitulemused}
\label{eelteadmised:skalaar}

\begin{teoreem} \textnormal{\cite[teoreem II.7.3]{FA2}}
Lõplikumõõtmelises skalaarkorrutisega $g$ varustatud vektorruumis $\mathbb{V}$ leidub ortonormeeritud baas.
\end{teoreem}
\begin{proof}
Esiteks märgime, et igas ühemõõtmelises vektorruumis eksisteerib ortonormeeritud baas, sest kui $\left\lbrace b \right\rbrace$ on mingi baas, siis $\left\lbrace \frac{1}{|b|} b\right\rbrace$ on ortonormeeritud baas.\\
Eeldame nüüd, et igas $(n-1)$-mõõtmelises vektorruumis on olemas ortonormeeritud baas ning olgu $\mathbb{V}$ $n$-mõõtmeline vektorruum baasiga $\left\lbrace b_1, b_2, \dots, b_n \right\rbrace$. Eelduse järgi on ruumis $\mathbb{V}$ ortonormeeritud süsteem $\left\lbrace e_1, e_2, \dots, e_{n-1} \right\rbrace$, kusjuures
\[ \spn\{ e_1, e_2, \dots, e_{n-1}\} = \spn\{ b_1, b_2, \dots, b_{n-1}\}. \]
Seega tarvitseb meil leida veel $a_n \in \mathbb{V} \setminus \{0\}$ omadusega
\[a_n \perp \spn\{ e_1, e_2, \dots, e_{n-1}\},\]
sest siis $\{ e_1, e_2, \dots, e_{n-1}, \frac{1}{|a_n|}a_n\}$ on ruumi $\mathbb{V}$ ortonormeeritud baas.\\
Otsime vektorit $a_n$ kujul
\begin{equation} \label{eq:otsitav-a_n}
a_n = b_n + \sum_{j = 1}^{n-1} \alpha^j e_j, \text{ kus } \alpha^1, \dots, \alpha^{n-1} \in \R.
\end{equation}
Paneme tähele, et kui $a_n$ on sellisel kujul, siis $a_n \neq 0$, sest vastasel korral $b_n \in \spn\{ b_1, b_2, \dots, b_{n-1}\}$, mis on vastuolus süsteemi $\spn\{ b_1, b_2, \dots, b_{n-1}\}$ lineaarse sõltumatusega.
Kui $a_n$ on kujul \ref{eq:otsitav-a_n}, siis kõikide $k \in \{1, 2, \dots, n-1\}$ korral
\begin{equation*}
a_n \perp e_k \iff a_n \cdot e_k = 0 \iff \left(b_n + \sum_{j = 1}^{n-1} \alpha^j e_j\right) \cdot e_k = 0
\end{equation*}
Samas, kuna
\begin{equation*}
\left(b_n + \sum_{j = 1}^{n-1} \alpha^j e_j\right) \cdot e_k = b_n \cdot e_k + \sum_{j = 1}^{n-1} \alpha^j \left(e_j \cdot e_k \right) = b_n \cdot e_k + \alpha_k,
\end{equation*}
siis $a_n \perp e_k \iff \alpha_k = - \left(b_n \cdot e_k \right)$. \newline
Järelikult võime võtta $a_n := b_n - \sum_{j=1}^{n-1}\left(b_n \cdot e_j\right)e_j$.
\end{proof}
\begin{markus} \label{markus:gram-schmidt}
Teoreemi 4.1 tõestuses antud algortimi ortonormeetirud baasi leidmiseks nimetatakse \emph{Gram-Schmidti algoritmiks} või \emph{ortogonaliseerimisprotsessiks}.
\end{markus}

\begin{teoreem}[Cauchy-Schwartz-Bunjakowski võrratus] \textnormal{\cite[teoreem II.1.1.]{FA2}} \label{teoreem:C-S-B}
Olgu $\mathbb{V}$ vektroruum skalaarkorrutisega $g : \mathbb{V} \times \mathbb{V} \rightarrow \R$. Sellisel juhul kehtib võrratus
\begin{equation} \label{eq:C-S-B}
g^2 \left(u, v \right) \leq g \left(u, u \right) g \left(v, v \right)
\end{equation}
kõikide $u, v \in \mathbb{V}$ korral. Seejuures võrdus kehtib parajasti siis, kui elemendid $u$ ja $v$ on lineaarselt sõltuvad.
\end{teoreem}

\begin{proof}
Olgu $\mathbb{V}$ reaalne vektorruum skalaarkorrutisega $g$ ning olgu $u, v \in \mathbb{V}$. Siis iga $\lambda \in \R$ korral
\begin{align*}
0 &\leq g\left(u+\lambda v,u+\lambda v\right) = g\left(u,u\right) + 2g\left(u, \lambda v \right) + g\left(\lambda v, \lambda v\right) = \\
&= g\left(u,u\right) + 2\lambda g\left(u, v \right) + \lambda^2 g\left(v,v\right) \leq g\left(u,u\right) + 2\lambda | g\left(u, v \right)| + \lambda^2 g\left(v,v\right).
\end{align*}
Saime $\lambda$ suhtes võrratuse
\begin{equation*}
g\left(v,v\right)\lambda^2 + 2|g\left(u, v \right)|\lambda + g\left(u,u\right) \geq 0,
\end{equation*}
mille reaalarvuliste lahendite hulk on $\R$. Kui $g\left(v,v\right) > 0$, siis on tegu ruut\-võrratusega. Seega vastava ruutvõrrandi diskriminandi jaoks kehtib
\begin{equation*}
4|g\left(u,v\right)|^2 - 4g\left(u,u\right)g\left(v,v\right) \leq 0,
\end{equation*}
millest järeldub vahetult võrratus \ref{eq:C-S-B}. Juhul $g\left(v,v\right) = 0$ peab kõikide $\lambda \in \R$ korral kehtima $2|g\left(u,v\right)|\lambda + g\left(u,u\right) \geq 0$, mis on võimalik vaid siis, kui $g\left(u,v\right) = 0$. Sellisel juhul on tingimuse \ref{eq:C-S-B} kehtivus aga ilmne.
\paragraph{}
Veendume veel, et võrratuses \ref{eq:C-S-B} kehtib võrdus parajasti siis, kui $u$ ja $v$ on lineaarselt sõltuvad. \\
Oletame esiteks, et vektorid $u$ ja $v$ on lineaarselt sõltuvad. Siis leidub $\alpha \in \R$ selliselt, et $u = \alpha v$. Seega
\begin{align*}
g^2\left(u,v\right) &= g^2 \left(\alpha v,v\right) = \alpha^2 g^2 \left(v,v\right) = \alpha^2 g \left(v,v\right) g \left(v,v\right) \\
&= g \left(\alpha v,\alpha v\right)g \left( v,v\right) = g \left(u,u\right)g \left(v,v\right),
\end{align*}
nagu tarvis.
\newline
Kehtigu nüüd tingimuses \ref{eq:C-S-B} võrdus. Veendume, et siis $u$ ja $v$ on lineaarselt sõltuvad. Üldistust kitsendamata võime eeldada, et $u \neq 0$ ja $v \neq 0$. Siis ka $g \left(u,u\right) \neq 0$ ja $g \left(v,v\right) \neq 0$. Paneme tähele, et
\[g^2 \left(u, v \right) = g \left(u, u \right) g \left(v, v \right)\]
on eelnevat arvestades samaväärne tingimusega
\[\frac{g^2 \left(u, v \right) g \left(v, v \right)}{g^2 \left(v, v \right) } = g \left(u, u \right). \]
Tähistades $a := \frac{g \left(u, v \right) }{g \left(v, v \right) }$, saame, et $a^2 g \left(v, v \right) = g \left(u, u \right)$ ehk $g \left(av, av \right) = g \left(u, u \right)$, millest $u = av$.
\end{proof}

\subsection{Tulemusi lineaaralgebrast} \label{eelteadmised:algebra}
\begin{lause} \label{lemma:ort-skalaar-on-isomorfism}
Olgu $\mathbb{V}$ vektorruum ja $L : \mathbb{V} \rightarrow \mathbb{V}$ lineaarne teisendus ning $g$ skalaarkorrutis ruumil $\mathbb{V}$. Kui $g \left(x, y\right) = g \left(Lx, Ly\right)$ kõikide $x, y \in \mathbb{V}$ korral, siis $L$ on isomorfism ruumil $\mathbb{V}$.
\end{lause}

\begin{proof}
Olgu $\mathbb{V}$ vektorruum ja $L : \mathbb{V} \rightarrow \mathbb{V}$ lineaarne teisendus ja kehtigu $g \left(x, y\right) = g \left(Lx, Ly\right)$ kõikide $x, y \in \mathbb{V}$ korral. Veendumaks, et $L$ on isomorfism piisab näidata, et $L$ on injektiivne ja sürjektiivne. Veendume kõigepealt kujutuse $L$ üksühesuses.\\
Olgu $x, y \in \mathbb{V}$, $x \neq y$. Oletame vastuväiteliselt, et $Lx = Ly$, siis $Lx - Ly = 0$ ja seega iga $z \in \mathbb{V}$ korral
\begin{equation*}
g \left( Lx - Ly, Lz \right) = 0.
\end{equation*}
Teisalt, kuna $g$ on skalaarkorrutis ja et $x \neq y$, siis leidub selline $z' \in \mathbb{V}$, et $g \left( x - y, z' \right) \neq 0$. Kokkuvõttes saime
\begin{equation*}
0 \neq g \left (x-y,z'\right ) = g\left (Lx - Ly, Lz'\right ) = 0,
\end{equation*}
mis on vastuolu. \\
Veendume nüüd teisenduse $L$ sürjektiivsuses. Olgu $x \in \mathbb{V}$. Meie eesmärk on leida $y \in \mathbb{V}$ selliselt, et $Ly = x$. Tähistame teisenduse $L$ maatriksi tähega $\Lambda$. Siis
\[Ly = L\left(y^a e_a\right) = y^a \Lambda\indices{^b_a}e_b = x^b e_b, \text{ kus } a,b = 1,2,3,4.\]
Saime võrrandisüsteemi $y^a \Lambda\indices{^b_a} = x^b$, mis on üheselt lahenduv, kuna $\det \Lambda = \pm 1$. Seega võime võtta $y = (y_1, y_2, y_3, y_4)$.
\end{proof}

\begin{teoreem} \label{teor:trace}
Olgu $A$ ja $B$ $n$-järku ruutmaatriksid. Siis $\Tr AB = \Tr BA$.
\end{teoreem}
\begin{proof}
Vahetu kontroll.
\end{proof}

\begin{teoreem}[Blokkmaatriksite korrutamine]
Olgu meil $m \times p$ maatriks $A$ ja olgu meil $p \times n$ maatriks $B$. Jagame maatriksi $A$ blokkideks, kus on $q$ rea blokki ja $s$ veeru blokki ning jagame maatriksi $B$ blokkideks, kus on $s$ rea blokki ja $r$ veeru blokki kujul
\begin{align*}
A = \begin{pmatrix}
A_{11} & A_{12} & \dots & A_{1s} \\
A_{21} & A_{22} & \dots & A_{2s} \\
\vdots & \vdots & \ddots & \vdots \\
A_{q1} & A_{q2} & \dots & A_{qs} \\
\end{pmatrix} \text{ ja }
B = \begin{pmatrix}
B_{11} & B_{12} & \dots & B_{1r} \\
B_{21} & B_{22} & \dots & B_{2r} \\
\vdots & \vdots & \ddots & \vdots \\
B_{s1} & B_{s2} & \dots & B_{sr} \\
\end{pmatrix}.
\end{align*}
Siis saame arvutada $m \times n$ maatriksi $C = AB$, kus on $q$ rea blokki ja $r$ veeru blokki kujul
\begin{align*}
C = \begin{pmatrix}
C_{11} & C_{12} & \dots & C_{1r} \\
C_{21} & C_{22} & \dots & C_{2r} \\
\vdots & \vdots & \ddots & \vdots \\
C_{q1} & C_{q2} & \dots & C_{qr} \\
\end{pmatrix}
\end{align*}
ning $C_{ab} = \sum_{i = 1}^{s} A_{ai} B_{ib}$.
\end{teoreem}

\begin{teoreem}[Blokkmaatriksi pöördmaatriks]
Olgu $M$ regulaarne maatriks, mis on esitatud blokkmaatriksina kujul $M = \begin{pmatrix} A & B \\ C & D \\ \end{pmatrix}$, kus $A, B, C, D$ on suvalise suurusega alammaatriksid, kusjuures $A$ ja $D$ on ruutmaatriksid. Siis saame maatriksi $M$ pöördmaatriksi $M^{-1}$ arvutada blokkidena järgmise eeskirja alusel:
\begin{align*}
M^{-1} = \begin{pmatrix}
A & B \\ C & D \\
\end{pmatrix}^{-1} = \qquad\qquad\qquad\qquad\qquad \\
= \begin{pmatrix}
A^{-1}+A^{-1}B\left(D-CA^{-1}B\right)^{-1}CA^{-1} & -A^{-1}B\left(D-CA^{-1}B\right)^{-1} \\
-\left(D- CA^{-1}B\right)^{-1}CA^{-1} & \left(D-CA^{-1}B\right)^{-1}
\end{pmatrix}.
\end{align*}
\end{teoreem}

\subsubsection{Maatrikseksponentsiaal}

Olgu $A \in \Mat_n\K$ mingi ruutmaatriks. Analoogilisest analüüsist tuttavale eksponentfunktsisoonile $e^x$ defineeritakse ka \emph{maatrikseksponent} $e^A = \sum_{k = 0}^{\infty} \frac{A^k}{k!}$, kus $A^0 = E$ ja $A^k$ tähendab korrutist, kus maatriksit $A$ korrutatakse iseendaga $k$ korda. Osutub, et $e^A$ on selles mõttes korrektselt defineeritud, et rida $\sum_{k = 0}^{\infty} \frac{A^k}{k!}$ koondub alati. Sageli kasutatakse $e^A$ asemel ka tähistust $\exp\left(A\right)$.\\
Järgnevas lauses toome mõned lihtsamad maatrikseksponendi omadused.
\begin{lause}
Olgu $X, Y \in \Mat_n \K$ ruutmaatriksid ja $k, l \in \K$. Siis
\begin{itemize}
\item $e^0 = E$,
\item $e^{aX}e^{bX} = e^{\left(a+b\right)X}$,
\item $e^{X} e^{-X} = E$,
\item kui $XY = YX$, siis $e^Y e^X = e^X e^Y = e^{\left(X + Y\right)}$,
\item kui $Y$ on pööratav, siis $e^{YXY^{-1}} = Ye^XY^{-1}$,
\item kui $e^X$ on pööratav, siis $\left(e^X\right)^{-1} = e^{-X}$.
\end{itemize}
\end{lause}

\subsection{Topoloogiline muutkond}

\begin{definitsioon}
Topoloogilist ruumi $\left(X, \tau\right)$ nimetatakse \emph{$n$-mõõtmeliseks topoloogilseks muutkonnaks}, kui
\begin{itemize}
\item[1)] $\left(X, \tau\right)$ on Hausdorffi topoloogiline ruum,
\item[2)] topoloogial $\tau$ leidub loenduv baas,
\item[3)] leidub hulga $X$ homöomorfism ruumi $\R^n$ lahtise alamhulgaga.
\end{itemize}
Muutkonda tähistame tähega $M$.
\end{definitsioon}

Muutkonda võime vaadelda kui geomeetrilise pinna üldistust. Vahetult muutkonna definitsiooni põhjal on selge ($\tau$ on Hausdorffi topoloogiline ruum), et muutkonna iga punkti mingit ümbrust võime vaadelda kui $n$-mõõtmelist eukleidilist ruumi.

\begin{definitsioon}
Olgu $M$ muutkond ja $B \subset \R^n$ lahtine alamhulk. Paari $\left(p, \psi\right)$, kus $p \in A \subset M$ ja $\psi : A \rightarrow B$ on homöomorfism nimetatakse muutkonna $M$ \emph{lokaalseks kaardiks} punktis $p$.\\
Kui muutkonna $M$ saab katta ühe lokaalse kaardiga, siis öeldakse, et muutkond $M$ on \emph{triviaalne}.
\end{definitsioon}

\begin{naide}
Järgmised hulgad on topoloogilised muutkonnad.
\begin{itemize}
\item[(a)] Ringjoon on $1$-mõõtmeline mittetriviaalne topoloogiline muutukond.
\begin{itemize}
\item[1)] Võime võtta topoloogiaks alamruumi topoloogia, see on Hausdorffi topoloogiline ruum.
\item[2)] Alamruumi topoloogial on samuti loenduv baas.
\item[3)] Homöomorfismiks sobib võtta neli lokaalset kaarti, mille saame, kui proijtseerime ülemise ja alumise poolringi $x$-teljele ja vasak- ning parempoolse poolringi $y$-teljele.
\end{itemize}
\item[(b)] Sfäär on topoloogiline muutkond. (Põhjendus on analoogline osaga (a).)
\end{itemize}
\end{naide}

\begin{definitsioon}
Me ütleme, et muutkond $M$ on \emph{ühelisidus}, kui suvalise kinnise joone võib pideva deformatsiooni abil tõmmata üheks punktiks.
\end{definitsioon}

\newpage

\section{Minkowski ruumi geomeetriline struktuur}

\subsection{Skalaarkorrutise definitsioon ja omadused}

Olgu $\mathbb{V}$ $n$-mõõtmeline vektorruum üle reaalarvude korpuse. Me ütleme, et kujutus $g : \mathbb{V} \times \mathbb{V} \rightarrow \R$ on \emph{bilineaarvorm}, kui $g$ on mõlema muutuja järgi lineaarne, see tähendab $g \left( \alpha_1 u_1 + \alpha_2 u_2, v \right) = \alpha_1 g \left( u_1, v \right) + \alpha_2 g \left( u_2, v \right)$ ja $g \left( u, \alpha_1 v_1 + \alpha_2 v_2 \right) = \alpha_1 g \left( u, v_1 \right) + \alpha_2 g \left( u, v_2 \right)$ kus $\alpha_1$ ja $\alpha_2$ on suvalised reaalarvud ning $u, u_1, u_2, v, v_1$ ja $v_2$ on vektorruumi $\mathbb{V}$ elemendid. 
\\
Olgu $u, v \in \mathbb{V}$. Bilineaarvormi $g$ nimetatakse \emph{sümmeetriliseks}, kui $g \left( u, v \right) = g \left(v, u \right)$ ja \emph{mittekidunuks}, kui tingumusest $g \left( u, v \right) = 0$ iga $v \in \mathbb{V}$ korral järeldub  $u = 0$.

\begin{definitsioon}
Mittekidunud sümmeetrilist bilineaarvormi $g: \mathbb{V} \times \mathbb{V} \rightarrow \R$ nimetatakse \emph{skalaarkorrutiseks}. Vektorite $u$ ja $v$ skalaarkorrutist tähistame sageli ka kujul $u \cdot v$.
\end{definitsioon}

Tänu skalaarkorrutise bilineaarsusele on kergesti tuletatavad järgmised omadused:
\begin{itemize}
\item $u \cdot 0 = 0 \cdot v = 0$ kõikide $u, v \in \mathbb{V}$ korral, sest $g$ bilineaarsuse tõttu $g \left(0, v\right) = g \left(0 \cdot 0, v \right) = 0 \cdot g \left(0, v \right) = 0$,
\item kui $u_1, u_2, \dots, u_n, u, v_1, v_2, \dots, v_n \in \mathbb{V}$, siis $\left( \sum_{i = 1}^{n} u_i \right) \cdot v = \sum_{i = 1}^{n}  \left( u_i \cdot v \right)$ ja $u \cdot \left( \sum_{i = 1}^{n} v_i \right) = \sum_{i = 1}^{n}  \left( u \cdot v_i \right)$,
\item kui $\left\lbrace e_1, e_2, \dots, e_n \right\rbrace$ on vektorruumi $\mathbb{V}$ baas ning kui tähistame $\eta_{ij} = e_i \cdot e_j$, $i,j = 1, 2, \dots, n$, siis $u \cdot v = \sum_{i = 1}^{n} \sum_{j = 1}^{n} \eta_{ij} u^i v^j = \eta_{ij} u^i v^j$, kus $u = u^i e_i$ ja $v = v^i e_i$.
\end{itemize}

\begin{naide}
Vaatleme ruumi $\R^{n}$. Olgu $u = \left(u^1, u^2, \dots, u^n \right), v = \left(v^1, v^2, \dots, v^n \right) \in \R^{n}$. Lihtne on veenduda, et kujutus $g \left(u, v \right) = u^1v^1 + u^2v^2 + \dots + u^n v^n$ on skalaarkorrutis.
\end{naide}

Näites 1 defineeritud skalaarkorrutis on \emph{positiivselt määratud}, see tähendab iga $v \neq 0$ korral $g \left(v, v \right) > 0$. Kui $g \left(v, v \right) < 0$ kõikide $v \neq 0$ korral, siis ütleme, et $g$ on \emph{negatiivselt määratud} ja kui $g$ pole ei positiivselt ega negatiivselt määratud, siis öeldakse, et $g$ on \emph{määramata}.

\begin{definitsioon}
Kui $g$ on skalaarkorrutis vektorruumil $\mathbb{V}$, siis nimetame vektoreid $u$ ja $v$ \emph{$g$-ortogonaalseteks} (või lihtsalt \emph{ortogonaalseteks}, kui $g$ roll on kontekstist selge), kui $g \left( u, v \right) = 0$ . Kui $\mathbb{W} \subset \mathbb{V}$ on alamruum, siis ruumi $\mathbb{W}$ ortogonaalne täiend $\mathbb{W}^{\perp}$ on hulk $\mathbb{W}^{\perp} = \left\lbrace u \in \mathbb{V} : \forall v \in  \mathbb{W} \text{ korral } g \left(u, v \right) = 0 \right\rbrace$.
\end{definitsioon}
\begin{definitsioon}
Skalaarkorrutise $g$ poolt määratud \emph{ruutvormiks} nimetame kujutust $Q : \mathbb{V} \rightarrow \R$, kus $Q \left( v \right) = g\left(v, v\right) = v \cdot v$, $v \in \mathbb{V}$.
\end{definitsioon}

\begin{lause} \label{lause:skalaarkorrutise-yhesus}
Olgu $g_1$ ja $g_2$ kaks skalaarkorrutist vektorruumil $\mathbb{V}$, mis rahuldavad tingimust $g_1 \left(u, u \right) = g_2 \left(u, u \right)$ iga $v \in \mathbb{V}$ korral. Siis kehtib $g_1 \left(u, v \right) = g_2 \left(u, v \right)$ kõikide $u, v \in \mathbb{V}$ korral, ehk teisi sõnu, $g_1 \equiv g_2$.
\end{lause}

% SIIA TULEB PÄRIS TÕESTUS KIRJUTADA!!!
\begin{proof}
Olgu $\mathbb{V}$ vektorruum ning olgu $u, v \in \mathbb{V}$ ja kehtigu võrdus $g_1 \left(u, u\right) = g_2 \left(u, u\right)$ iga $u$ korral. Defineerime uue kujutuse
\begin{equation*}
g: \mathbb{V} \times \mathbb{V} \rightarrow \R \text{, } g\left(u, v\right) \mapsto g_1\left(u, v\right) - g_2\left(u, v\right).
\end{equation*}
Paneme esiteks tähele, et selliselt defineeritud $g$ on sümmeetriline ja bilineaarne. Tõepoolest, olgu $u_1, u_2 \in \mathbb{V}$. Siis
\begin{align*}
g\left( \alpha u_1 + \beta u_2, v \right) &= g_1\left( \alpha u_1 + \beta u_2, v \right) - g_2\left( \alpha u_1 + \beta u_2, v \right) = \\
&= \alpha g_1 \left(u_1, v\right) + \beta g_1 \left(u_2, v \right) -\alpha g_2 \left(u_1, v\right)  \beta g_2 \left(u_2, v \right) = \\
&= \alpha \left( g_1 \left(u_1, v\right) - g_2 \left(u_1, v \right)\right) + \beta \left( g_1 \left(u_2, v\right) - g_2 \left(u_2, v \right)\right) = \\
&= \alpha g\left( u_1, v \right) + \beta g\left( u_2, v\right) \text{ ja analoogiliselt} \\
g\left( u, \alpha v_1 + \beta v_2 \right) &= \alpha g\left( u, v_1 \right) + \beta g\left( u, v_2\right).
\end{align*}
Kujutuse $g$ sümmeetrilisus on $g_1$ ja $g_2$ sümmeetrilisust arvestades ilmne. Tõestuse lõpetamiseks piisab nüüd näidata, et $g = 0$. Ühelt poolt paneme tähele, et
\begin{equation*}
g \left(u+v, u+v\right) = g_1 \left(u+v, u+v\right) - g_2 \left(u+v, u+v\right) = 0.
\end{equation*}
Teisalt,
\begin{align*}
g \left(u+v, u+v\right) &= g \left(u, u+v\right) + g \left(v, u+v\right) = \\
&= g \left(u, u\right) + g \left(u, v\right) + g \left(v, u\right) + g \left(v, v\right) = \\
&= g \left(u, u\right) + 2g \left(u, v\right) + g \left(v, v\right) = 2g \left(u, v\right).
\end{align*}
Kokkuvõttes saime $2g\left(u, v\right) = 0$ ehk $g\left(u, v\right) = 0$, mida oligi tarvis.
\end{proof}

\begin{teoreem}
Olgu $\mathbb{V}$ reaalne $n$-mõõtmeline vektorruum ning olgu $g : \mathbb{V} \times \mathbb{V} \rightarrow \R$ skalaarkorrutis. Vektorruumil $\mathbb{V}$ leidub baas $\left\lbrace e_1, e_2, \dots, e_n \right\rbrace$ nii, et $g \left(e_i, e_j\right) = 0$ kui $i \neq j$ ja $Q\left(e_i\right) = \pm 1$ iga $i = 1, 2, \dots, n$ korral. Enamgi veel, baasivektorite arv, mille korral $Q \left(e_i\right) = -1$ on sama kõikide neid tingimusi rahuldavate baaside korral sama.
\end{teoreem}

\begin{proof}
Arvestades \textit{Gram\footnote{Jørgen Pedersen Gram (1850 – 1916) - taani matemaatik}-Schmidti\footnote{Erhard Schmidt (1876 – 1959) - Tartus sündinud saksa matemaatik}} algoritmi ortonormeeritud baasi leidmiseks, muutub teoreemi tõestus ilmseks\footnote{Vaata \ref{eelteadmised:skalaar} \nameref{eelteadmised:skalaar}, Märkus \ref{markus:gram-schmidt}}.
% Lisa viide Gram-Schmidti algoritmile
\end{proof}

\begin{definitsioon}
Vektorruumi $\mathbb{V}$ baasi teoreemist 4.2 nimetame ortonormeeritud baasiks.
\end{definitsioon}

Skalaarkorrutise $g$ suhtses ortonormaalse baasi $\left\lbrace e_1, e_2, \dots, e_n \right\rbrace$ vektorite arvu $r$, mille korral $Q \left(e_i\right) = -1, i \in \left\lbrace 1, 2, \dots, n \right\rbrace$, nimetame skalaarkorrutise $g$ \emph{indeksiks}.
Edasises eeldame, et ortonormeeritud baasid on indekseeritud nii, et baasivektorid $e_i$, mille korral $Q \left(e_i\right) = -1$, paiknevad loetelu lõpus, ehk ortonormeeritud baasi 
\[\left\lbrace e_1, e_2, \dots, e_{n-r}, e_{n-r+1}, \dots, e_n \right\rbrace\]
korral $Q \left(e_i\right) = 1$, kui $i = 1, 2, \dots, n-r$, ja $Q \left(e_i\right) = -1$, kui $i = n-r+1, \dots, n$. Tähistades $u = u^i e_i$ ja $v = v^i e_i$ saame sellise baasi suhtes skalaarkorrutise $g$ arvutada järgmiselt:
\[g\left(u, v\right) = u^1 v^1 + u^2 v^2 + \dots + u^{n-r} v^{n-r} - u^{n-r+1} v^{n-r+1} - \dots - u^n v^n.\]

\begin{markus}
Vektorruumi $\mathbb{V}$ skalaarkorrutisega $g$, mille indeks $r > 0$ nimetatakse \emph{pseudoeukleidiliseks ruumiks}.
\end{markus}


\subsection{Minkowski aegruumi mõiste}

\begin{definitsioon}
\emph{Minkowski aegruumiks} nimetatakse $4$-mõõtmelist reaalset vektorruumi $\M$, millel on defineeritud mittekidunud sümmeetriline bilineaarvorm g indeksiga $1$. \\
Ruumi $\M$ elemente nimetatakse \emph{sündmusteks} ja kujutust $g$ nimetatakse \emph{Lorentzi skalaarkorrutiseks} ruumil $\M$.
\end{definitsioon}

\begin{markus}
Edasises ütleme Minkowski ruumi kontekstis Lorentzi skalaarkorrutise $g$ kohta lihtsalt skalaarkorrutis.
\end{markus}

Ilmselt on Minkowski ruum pseudoeukleidiline ruum. Vahetult Minkowski ruumi definitsioonist selgub, et ruumil $\M$ leidub baas $\{e_1, e_2, e_3, e_4\}$ järgmise omadusega. Tähistades $u = u^i e_i$ ja $v = v^i e_i$, siis
\[g\left(u, v\right) = u^1 v^1 + u^2 v^2 + u^3 v^3 - u^4 v^4.\]

Olgugi $\{e_1, e_2, e_3, e_4\}$ või lühidalt $\{e_a\}$ ruumi $\M$ ortonormeeritud baas. 
Kui $x = x^1 e_1 + x^2 e_2 + x^3 e_3 + x^4 e_4$, siis tähistame sündmuse $x$ koordinaadid baasi $\{e_a\}$ suhtes $\left( x^1, x^2, x^3, x^4 \right)$ ja seejuures ütleme, et $\left( x^1, x^2, x^3 \right)$ on \emph{ruumikoordinaadid} ning $\left(x^4\right)$ on \emph{ajakoordinaat}.
\paragraph{}
Kuna Lorentzi skalaarkorrutis $g$ ei ole ruumil $\M$ positiivselt määratud, siis leiduvad vektorid $u \in \M \setminus \{0\}$ nii, et $g \left(u, u\right) = 0$. Selliseid vektoreid nimetatakse \emph{nullpikkusega vektoriteks}. Kui aga $g \left(u, u\right) < 0$, siis ütleme, et $u$ on \emph{ajasarnane}\footnote{inglise keeles \textit{timelike}} ning kui $g \left(u, u\right) > 0$, siis nimetame vektorit $u$ \emph{ruumisarnaseks}\footnote{inglise keeles \textit{spacelike}}. Osutub, et ruumis $\M$ leidub koguni baase, mis koosnevad vaid nullpikkusega vektoritest.

\begin{naide}
Üheks ruumi $\M$ baasiks, mis koosneb vaid nullpikkusega vektoritest on näiteks $\{e_1^0, e_2^0, e_3^0, e_4^0\}$, kus $e_1^0 = \left(1, 0, 0, 1\right)$, $e_2^0 = \left(0, 1, 0, 1\right)$, $e_3^0 = \left(0, 0, 1, 1\right)$ ja $e_4^0 = \left(-1, 0, 0, 1\right).$
Tõepoolest, süsteemi $\{e_1^0, e_2^0, e_3^0, e_4^0\}$ lineaarne sõltumatus on vahetult kontrollitav ja $e_1^0, \dots, e_4^0$ on nullpikkusega, sest
\begin{align*}
Q\left(e_1^0\right) &= 1^2 + 0 + 0 - 1^2 = 0, \\
Q\left(e_2^0\right) &= 0 + 1^2 + 0 - 1^2 = 0, \\
Q\left(e_3^0\right) &= 0 + 0 + 1^2 - 1^2 = 0, \\
Q\left(e_4^0\right) &= (-1)^2 + 0 + 0 - 1^2 = 0.
\end{align*}
\end{naide}

Samas paneme tähele, et selline baas ei saa koosneda paarikaupa ortogonaalsetest vektoritest.
\begin{teoreem}
Olgu $u, v \in \M \setminus \{0\}$ nullpikkusega vektorid. Vektorid $u$ ja $v$ on ortogonaalsed siis ja ainult siis, kui nad on paralleelsed, st leidub $t \in \R$ nii, et $u = tv$.
\end{teoreem}
\begin{proof}
\emph{Piisavus.} Olgu $u, v \in \M \setminus \{0\}$ paralleelsed nullpikkusega vektorid. Siis leidub $t \in \R$ nii, et $u = tv$. Seega
\[g\left(u, v\right) = g \left(tv, v\right) = t g \left(v, v\right) = 0\]
ehk vektorid $u$ ja $v$ on ortogonaalsed, nagu tarvis.
\\
\emph{Tarvilikkus.} Olgu $u, v \in \M \setminus \{0\}$ ortogonaalsed nullpikkusega vektorid, st $g \left(u, v\right) = 0$. \emph{Cauchy-Schwartz-Bunjakowski võrratuse}\footnote{Vaata \ref{eelteadmised:skalaar} \nameref{eelteadmised:skalaar}, Teoreem \ref{teoreem:C-S-B}} $g^2 \left(u, v \right) \leq g \left(u, u \right) g \left(v, v \right)$ põhjal $0 \leq g \left(u, u \right) g \left(v, v \right)$, sest $u$ ja $v$ on ortogonaalsed. Teisalt, et $u$ ja $v$ on nullpikkusega vektorid, siis $g \left(u, u \right) g \left(v, v \right) = 0$ ja järelikult kehtib Cauchy-Schwartz-Bunjakowski võrratuses võrdus $0 = 0$, mis tähendab, et $u$ ja $v$ on lineaarselt sõltuvad.
\end{proof}

Võtame nüüd vaatluse alla kaks sellist sündmust $x, x_0 \in \M$, $x \neq x_0$, mida ühendab nullpikkusega vektor, see tähendab $Q \left(x- x_0\right) = 0$. Seda asjaolu arvesse võttes saame, et kui $\{e_a\}$ ruumi $\M$ ortonormaalne baas ja me tähistame $x = x^a e_a$, $x_0 = x_0^a e_a$, siis kehtib võrdus
\begin{equation} \label{eq:nullkoonus}
Q \left(x - x_0\right) = \left(x^1 - x_0^1\right)^2 + \left(x^2 - x_0^2\right)^2 + \left(x^3 - x_0^3\right)^2 - \left(x^4 - x_0^4\right)^2 = 0.
\end{equation}
Kõigi selliste $x \in \M$ hulka, mille korral on tingimus \ref{eq:nullkoonus} täidetud nimetatakse \emph{nullkoonuseks}\footnote{Füüsikas öeldakse sageli \emph{nullkoonuse} asemel \emph{valguse koonus}.} punktis $x_0$ ja tähistatakse $\mathcal{C}_N\left(x_0\right)$. Seega 
\[\mathcal{C}_N\left(x_0\right) = \{x \in \M : Q \left(x- x_0\right) = 0 \}.\]
Piltlikult võime öelda, et hulga $\mathcal{C}_N\left(x_0\right)$ elemendid on ühendatavad sündmusega $x_0$ \emph{valguskiire} $R_{x_0, x} = \{x_0 + t\left(x - x_0\right) : t \in \R \}$ abil.

\subsection{Ortogonaalteisendus ruumis $\M$}

Olgu $\left\lbrace e_1, e_2, e_3, e_4 \right\rbrace$ ja $\left\lbrace \hat{e}_1, \hat{e}_2, \hat{e}_3, \hat{e}_4 \right\rbrace$ ruumi $\M$ kaks ortonormaalset baasi. Osutub, et leidub parajasti üks selline lineaarne kujutus $L : \M \rightarrow \M$, et $L\left(e_a\right) = \hat{e}_a$, $a = 1, 2, 3, 4$. Tõepoolest, leiduvad arvud $\Lambda\indices{^a_b}$ nii, et baasi $\{\hat{e}_a\}$ vektorid avalduvad baasi $\{e_a\}$ suhtes üheselt kujul $\hat{e}_b = \Lambda\indices{^a_b} e_a$. Arvudest $\Lambda\indices{^a_b}$ tekkiva maatriksiga assotsieeruv Lorentzi teisendus sobibki otsitavaks teisenduseks $L$. Järgnevaga uurime kujutuse $L$ omadusi veidi lähemalt.

\begin{definitsioon}
Ruumi $\M$ lineaarset kujutust $L : \M \rightarrow \M$ nimetatakse \emph{pseudo\-ortogonaalteisenduseks}, kui ta säilitab skalaarkorrutise $g$, see tähendab iga $x$ ja $y$ korral ruumist $\M$ kehtib võrdus $g \left(Lx, Ly \right) = g \left(x, y\right)$.
\end{definitsioon}

\begin{lause} \label{lause:ortogonaalteisendus}
Olgu $L:\M \rightarrow \M$ lineaarne kujutus. Siis järgmised väited on samaväärsed:
\begin{itemize}
\item[(i)] $L$ on pseudoortogonaalteisendus;
\item[(ii)] $L$ säilitab ruumi $\M$ ruutvormi, see tähendab $Q\left(Lx\right) = Q\left(x\right)$ iga $x \in \M$ korral;
\item[(iii)] $L$ kujutab suvalise ruumi $\M$ ortonormaalse baasi ruumi $\M$ ortonormaalseks baasiks.
\end{itemize}
\end{lause} 

\begin{proof}
$(i) \implies (ii)$. Olgu $L$ pseudoortogonaalne teisendus. Siis definitsiooni põhjal $g\left(Lx, Ly\right) = g\left(x, y\right)$ iga $x, y \in \M$ korral. Seega kehtib ka $Q\left(Lx\right) = g\left(Lx, Lx\right) = g\left(x, x\right) = Q\left(x\right)$ kõikide $x \in \M$ korral ehk $L$ säilitab ruutvormi. \\
$(ii) \implies (i)$ on täpselt lause \ref{lause:skalaarkorrutise-yhesus}. \\
$(ii) \implies (iii)$. Kehtigu $(ii)$ (ja seega ka $(i)$) ning olgu $\left\lbrace e_1, e_2, e_3, e_4 \right\rbrace$ ortogonaalne baas ruumis $\M$. Siis ka $\left\lbrace Le_1, Le_2, Le_3, Le_4 \right\rbrace$ on ortonormaalne baas ruumis $\M$, sest
\begin{equation*}
g \left(Le_i, Le_j\right) = g\left(e_i, e_j\right) = \begin{cases}
    -1,& \text{kui $i = j = 4$},\\
    1,& \text{kui $i = j$, $i,j \in \{1, 2, 3\}$},\\
    0,& \text{kui $i \neq j$}.
  \end{cases}
\end{equation*}
ja arvestades kujutuse $L$ lineaarsust, on ka süsteem $\left\lbrace Le_1, Le_2, Le_3, Le_4 \right\rbrace$ lineaarselt sõltumatu. \\
$(iii) \implies (ii)$. Olgu $\left\lbrace e_1, e_2, e_3, e_4 \right\rbrace$ ruumi $\M$ ortonormeeritud baas ja kehtigu tingimus $(iii)$. Veendume, et alati $Q\left(Lx\right) = Q\left(x\right)$, kus $x \in \M$ on suvaline. Fikseerime $x \in \M$ ning esitugu ta koordinaatides kujul $x = x^i e_i$, $i = 1, 2, 3, 4$.
\begin{align*}
Q\left(x\right) &= Q\left(x^i e_i\right) = x^i Q\left(e_i\right) = x^i Q\left(Le_i\right) = \\
&= Q\left(x^i L e_i\right) = Q\left(L\left(x^i e_i\right)\right) = Q\left(Lx\right). \qedhere
\end{align*}
\end{proof}
%
Ruumi kokkuhoiu mõttes toome sisse $4 \times 4$ maatriksi $\eta$, mille me defineerime kui
\begin{equation*}
\eta = \begin{pmatrix}
1 & 0 & 0 & 0 \\ 
0 & 1 & 0 & 0 \\  
0 & 0 & 1 & 0 \\ 
0 & 0 & 0 & -1 \\ 
\end{pmatrix},
\end{equation*}
ja mille elemente tähistame vastavalt vajadusele kas $\eta_{ab}$ või $\eta^{ab}$, $a, b = 1, 2, 3, 4$. Loomulik on maatriksit $\eta$ nimetada \emph{Miknowski ruumi meetrikaks}.\\
Sellise tähistuse korral $\eta_{ab} = 1$, kui $a = b = 1, 2, 3$ ja $\eta_{ab} = -1$, kui $a = b = 4$ ning $\eta_{ab} = 0$ muudel juhtudel. Vahetult on kontrollitav, et $\eta = \eta^T$ ja $\eta \eta^{-1} = \eta^{-1} \eta = E$, kus $E$ on ühiksmaatriks.
\newline
Arvestades sissetoodud tähistusi saame nüüd kirjutada $g\left(e_a, e_b\right) = \eta_{ab}$, kus $\{e_a\}$ on ruumi $\M$ ortonormeeritud baas. Enamgi veel, avaldades vektorid $u, v \in \M$ baasivektorite kaudu $u = u^i e_i$ ja $v = v^i e_i$, saame summeerimiskokkulepet kasutades kirjutada $g \left(u, v\right) = \eta_{ab} u^a v^b$.

\paragraph{} Olgu $L : \M \rightarrow \M$ ruumi $\M$ pseudoortogonaalteisendus ja $\{e_1, e_2, e_3, e_4\}$ selle ruumi ortonormeeritud baas. Lause \ref{lause:ortogonaalteisendus} põhjal on siis ka $\hat{e}_1 = Le_1, \hat{e}_2  = Le_2, \hat{e}_3 = Le_3, \hat{e}_4 = Le_4$ ruumi $\M$ ortonormeeritud baas, kusjuures iga $e_i$, $i = 1,2,3,4$ saab esitada vektorite $\hat{e}_j$ lineaarkombinatsioonina kujul
\begin{equation} \label{eq:baasideseos}
e_i = \Lambda\indices{^1_i} \hat{e}_1 + \Lambda\indices{^2_i} \hat{e}_2 + \Lambda\indices{^3_i} \hat{e}_3 + \Lambda\indices{^4_i} \hat{e}_4 = \Lambda\indices{^j_i} \hat{e}_j, \quad i,j = 1, 2, 3, 4,
\end{equation}
kus arvud $\Lambda\indices{^j_i}$ on mingid reaalarvulised konstandid. Arvestades valemit \ref{eq:baasideseos} võime nüüd ortogonaalsuse tingimuse $g \left(e_c, e_d\right) = \eta_{cd}$, $c, d = 1, 2, 3, 4$, kirjutada kujul
\begin{equation*}
\Lambda\indices{^1_c} \Lambda\indices{^1_d} + \Lambda\indices{^2_c} \Lambda\indices{^2_d} + \Lambda\indices{^3_c} \Lambda\indices{^3_d} - \Lambda\indices{^4_c} \Lambda\indices{^4_d} = \eta_{cd}
\end{equation*}
või kasutades summeerimiskokkulepet, siis lühidalt
\begin{equation} \label{eq:lambdaeta1}
\Lambda\indices{^a_c} \Lambda\indices{^b_d} \eta_{ab} = \eta_{cd}.
\end{equation}
Seose \ref{eq:lambdaeta1} saame maatrikskujul kirjutada kui
\begin{equation} \label{eq:lambdaetamat1}
\Lambda^T \eta \Lambda = \eta.
\end{equation}
Korrutades võrduse \ref{eq:lambdaetamat1} mõlemaid pooli maatriksiga $\Lambda^{-1}$ saame $\Lambda^T \eta = \eta \Lambda^{-1}$. Korrutades nüüd saadud tulemust veel paremalt maatriksiga $\eta^{-1}$ on tulemuseks $\Lambda^T = \eta \Lambda^{-1} \eta^{-1}$. Viimast arvesse võttes võime kirjutada $\Lambda \eta \Lambda^T = \Lambda \eta^{-1} \Lambda^T = \Lambda \eta^{-1} \left( \eta \Lambda^{-1} \eta^{-1} \right) = \Lambda \Lambda^{-1} \eta^{-1} = \eta^{-1} = \eta$ ehk
\begin{equation} \label{eq:lambdaetamat2}
\Lambda \eta \Lambda^T = \eta.
\end{equation}
Seos \ref{eq:lambdaetamat2} on koordinaatides kirjutatuna täpselt
\begin{equation} \label{eq:lambdaeta2}
\Lambda\indices{^a_c} \eta^{cd} \Lambda\indices{^b_d} = \eta^{ab}.
\end{equation}
\begin{definitsioon}
Maatriksit $\Lambda = \left[\Lambda\indices{^a_b} \right]_{a,b=1,2,3,4}$ nimetame \emph{pseudoortogonaalteisendusega $L$ ja baasiga $\{e_a\}$} assotsieeruvaks maatriksiks.
\end{definitsioon}
Definitsioonile eelnevas arutelus tõestasime maatriksi $\Lambda$ kohta järgmise lemma.
\begin{lemma} \label{lemma:lambdainvariant}
Pseudoortogonaalteisendusega $L$ ja baasiga $\{e_a\}$ assotsieeruva maatriksi $\Lambda$ korral on tingimused \ref{eq:lambdaeta1}, \ref{eq:lambdaetamat1}, \ref{eq:lambdaetamat2} ja \ref{eq:lambdaeta2} samaväärsed.
\end{lemma}

Kuna ortogonaalteisenduse maatriks mistahes ortonormeeritud baasi suhtes on ortogonaalmaatriks, ja vastupidi, kui ortogonaalteisenduse maatriks mingi ortonormeeritud baasi suhtes on ortogonaalmaatriks, siis sellest järeldub kergesti järgmine lause.

\begin{lause}\textnormal{\cite[lk 271]{Kilp}}
Kui $\Lambda$ on ortogonaalteisendusega $L$ ja baasiga $\{e_a\}$ assotsieeruv maatriks, siis $\Lambda$ on ka ortogonaalteisenduse $L^{-1}$ ja baasiga $\{\hat{e}_a\} = \{Le_a\}$ assotsieeruv maatriks.
\end{lause}

Lihtne on veenduda, et lause väide jääb kehtimaka juhul kui asendame lause sõnastuses \emph{ortogonaalteisenduse} ja \emph{ortogonaalmaatriksi} vastavalt \emph{pseudoortogonaalteisenduse} ja \emph{pseudoortogonaalmaatriksiga}.\footnote{\cite{Kilp} antud tõestuses tuleb asendada eukleidiline meetrika $\delta_{ij}$ Minkowski meetrikaga $\eta_{ij}$ ja seega jääb tulemus kehtima ka pseudoortogonaalsel juhul.}
\paragraph*{}
Me vaatleme ortogonaalteisendusega $L$ ja baasiga $\{e_a\}$ seotud maatriksit $\Lambda$ kui koordinaatide teisenemise maatriksit tavalisel viisil. Seega, kui $x \in \M$ on esitub koordinaatides baasi $\{e_i\}$ suhtes kujul $x = x^i e_i$, $i = 1, 2, 3, 4$, siis tema koordinaadid baasi $\{\hat{e}_i\} = \{Le_i\}$ suhtes avalduvad kujul $x = \hat{x}^i \hat{e}_i$, kus
\begin{align*}
\hat{x}^1 &= \Lambda\indices{^1_1} x^1 + \Lambda\indices{^1_2} x^2 + \Lambda\indices{^1_3} x^3 + \Lambda\indices{^1_4} x^4, \\
\hat{x}^2 &= \Lambda\indices{^2_1} x^1 + \Lambda\indices{^2_2} x^2 + \Lambda\indices{^2_3} x^3 + \Lambda\indices{^2_4} x^4, \\
\hat{x}^3 &= \Lambda\indices{^3_1} x^1 + \Lambda\indices{^3_2} x^2 + \Lambda\indices{^3_3} x^3 + \Lambda\indices{^3_4} x^4, \\
\hat{x}^4 &= \Lambda\indices{^4_1} x^1 + \Lambda\indices{^4_2} x^2 + \Lambda\indices{^4_3} x^3 + 
\Lambda\indices{^4_4} x^4,
\end{align*}
mille võime lühidalt kirja panna kui
\begin{equation*}
\hat{x}^i = \Lambda\indices{^i_j} x^j \text{, kus } i,j=1, 2, 3, 4.
\end{equation*}

\begin{definitsioon}
$4 \times 4$ maatriksit $\Lambda$, mis rahuldab tingimust \ref{eq:lambdaetamat1} (ja lemma \ref{lemma:lambdainvariant} põhjal siis ka tingimusi \ref{eq:lambdaeta1}, \ref{eq:lambdaetamat2} ja \ref{eq:lambdaeta2}) nimetatakse \emph{(homogeenseks) Lorentzi teisenduseks}.
\end{definitsioon}

Kuna ruumi $\M$ ortogonaalsteisendus $L$ on isomorfism\footnote{{Vaata \ref{eelteadmised:algebra} \nameref{eelteadmised:algebra}, Lause \ref{lemma:ort-skalaar-on-isomorfism}}} 
ja seega pööratav, siis temaga assotsieeruv maatriks $\Lambda$ on samuti pööratav, kusjuures
\begin{equation} \label{eq:lambda-1lambda-trans}
\Lambda^{-1} = \eta \Lambda^T \eta.
\end{equation}
Tõepoolest, arvestades tingimust \ref{eq:lambdaetamat1} ja asjaolu, et $\eta = \eta^{-1}$, siis saame
\begin{equation*}
\eta = \Lambda^T \eta \Lambda \iff \eta \Lambda^{-1} = \Lambda^T \eta \iff \eta^{-1} \eta \Lambda^{-1} = \eta^{-1} \Lambda^T \eta \iff \Lambda^{-1} = \eta \Lambda^T \eta.
\end{equation*}

\begin{teoreem}
Kõigi (homogeensete) Lorentzi teisenduste hulk on rühm maatriksite korrutamise suhtes. Seda rühma nimetatakse \emph{(homogeenseks) Lorentzi rühmaks} ja tähistatakse $\L_{GH}$.
\end{teoreem}

\begin{proof}
Veendumaks, et kõigi (homogeensete) Lorentzi teisenduste hulk $\L_{GH}$ on rühm peame näitama, et $\L_{GH}$ on kinnine korrutamise ja pöörd\-elemendi võtmise suhtes.\\
Olgu $\Lambda, \Lambda_1, \Lambda_2 \in \L_{GH}$. Veendume esiteks, et korrutis $\Lambda_1 \Lambda_2$ kuulub hulka $\L_{GH}$. Selleks piisab näidata, et $\left(\Lambda_1 \Lambda_2\right)^T \eta \left(\Lambda_1 \Lambda_2\right) = \eta$.
\begin{equation*}
\left(\Lambda_1 \Lambda_2\right)^T \eta \left(\Lambda_1 \Lambda_2\right) = \left(\Lambda_2^T \Lambda_1^T\right) \eta \left(\Lambda_1 \Lambda_2\right) = \Lambda_2^T \left(\Lambda_1^T \eta \Lambda_1 \right) \Lambda_2 = \Lambda_2^T \eta \Lambda_2 = \eta,
\end{equation*}
ja seega $\Lambda_1 \Lambda_2 \in \L_{GH}$ nagu tarvis. \\
Jääb veel näidata, et ka $\Lambda^{-1 } \in \L_{GH}$. Seose \ref{eq:lambda-1lambda-trans} ja võrduste $\eta = \eta^T$, $\eta \eta = E$ põjal saame kirjutada
\begin{align*}
\left(\Lambda^{-1}\right)^T \eta \Lambda^{-1} &= \left(\eta \Lambda^T \eta \right)^T \eta \left(\eta \Lambda^T \eta\right) = \left(\eta^T \left(\Lambda^T\right)^T \eta^T \right) \eta \eta \Lambda^T \eta = \\
&= \eta \Lambda \eta \Lambda^T \eta = \eta \eta \eta = \eta.
\end{align*}
Viimane aga tähendabki, et $\Lambda^{-1} \in \L_{GH}$.
\end{proof}

Arvestades võrdust \ref{eq:lambda-1lambda-trans} võime välja arvutada maatriksi $\Lambda^{-1}$ ja esitada ta maatriksi $\Lambda$ elementide kaudu:
\[\Lambda^{-1} = \begin{pmatrix}
\Lambda\indices{^1_1} & \Lambda\indices{^2_1} & \Lambda\indices{^3_1} & -\Lambda\indices{^4_1} \\ 
\Lambda\indices{^1_2} & \Lambda\indices{^2_2} & \Lambda\indices{^3_2} & -\Lambda\indices{^4_2} \\  
\Lambda\indices{^1_3} & \Lambda\indices{^2_3} & \Lambda\indices{^3_3} & -\Lambda\indices{^4_3} \\ 
-\Lambda\indices{^1_4} & -\Lambda\indices{^2_4} & -\Lambda\indices{^3_4} & \Lambda\indices{^4_4} \\ 
\end{pmatrix}.\]

\newpage

\section{Lorentzi ja Poincar\'e rühmad}

\subsection{Lorentzi rühm}

\begin{definitsioon}
Me ütleme, et $\Lambda \in \L_{GH}$ on \emph{ortokroonne}, kui $\Lambda\indices{^4_4} \geq 1$ ja \emph{mitteortokroonne}, kui $\Lambda\indices{^4_4} \leq -1$.
\end{definitsioon}

Edasise teooriaarenduse seisukohalt on otstarbekas tõestada järgmine teoreem.
\begin{teoreem} \textnormal{\cite[teoreem 1.3.1]{Naber}} \label{teoreem:ajasarnased_vektorid}
Olgu $u, v \in \M$, kusjuures $u$ on ajasarnane ja $v$ ajasarnane või nullpikkusega ning olgu $\{e_a\}$ ruumi $\M$ ortonormaalne baas, mille suhtes $u$ ja $v$ avalduvad kujul $u = u^a e_a$ ja $v = v^a e_a$. Siis kehtib parajasti üks järgmistest tingimusest:
\begin{itemize}
\item[(a)] $u^4 v^4 > 0$, mille korral $g\left(u, v\right) < 0$,
\item[(b)] $u^4 v^4 < 0$, mille korral $g\left(u, v\right) > 0$.
\end{itemize}
\end{teoreem}
\begin{proof}
Oletame, et teoreemi eeldused on täidetud. Siis
\begin{align*}
g \left(u, u\right) &= \left(u^1\right)^2 + \left(u^2\right)^2 + \left(u^3\right)^2 - \left(u^4\right)^2 < 0 \text{ ja} \\
g \left(v, v\right) &= \left(v^1\right)^2 + \left(v^2\right)^2 + \left(v^3\right)^2 - \left(v^4\right)^2 \leq 0.
\end{align*}
Niisiis
\begin{align*}
\left(u^4 v^4\right)^2 &> \left( \left(u^1\right)^2 + \left(u^2\right)^2 + \left(u^3\right)^2 \right) \left( \left(v^1\right)^2 + \left(v^2\right)^2 + \left(v^3\right)^2 \right) \overset{(*)}{\geq} \\
&\geq u^1 v^1 + u^2 v^2 + u^3 v^3,
\end{align*}
kus võrratus $(*)$ tuleneb Cauchy-Scwartz-Bunjakowski võrratusest ruumi $\R^3$ jaoks.
Nüüd aga $|u^4 v^4| > |u^1 v^1 + u^2 v^2 + u^3 v^3|$, millest saame, et $u^4 v^4 \neq 0$ ja $g\left (u, v\right ) \neq 0$.
Oletame konkreetsuse mõttes, et $u^4 v^4 > 0$. Siis
\[u^4 v^4 = |u^4 v^4| > |u^1 v^1 + u^2 v^2 + u^3 v^3| \geq u^1 v^1 + u^2 v^2 + u^3 v^3,\]
millest
\[u^1 v^1 + u^2 v^2 + u^3 v^3 - u^4 v^4 < 0\]
ehk $g\left (u, v\right ) < 0$.
Kui aga $u^4 v^4 > 0$, siis $g\left (u, -v\right ) < 0$ ja seega $g \left (u, v\right ) > 0$.
\end{proof}

\begin{jareldus}
Kui $u \in \M \setminus \{0\}$ on ortogonaalne ajasarnase vektoriga $v \in \M$, siis $u$ on ruumisarnane.
\end{jareldus}
\begin{proof}
Olgu $v \in \M$ ajasarnane ja olgu $u \in \M$ nullist erinev ning ortogonaalne vektoriga $v$. Oletame vastuväiteliselt, et $u$ on ajasarnane. Siis eelmise teoreemi põhjal kehtib kas $g\left(u,v\right) > 0$ või $g\left(u,v\right) < 0$. Et $u$ ja $v$ on ortogonaalsed, siis $g\left(u,v\right) = 0$, mis on vastuolus eelnevaga. Järelikult on $u$ ruumisarnane.
\end{proof}

Tähistame ruumi $\M$ kõigi ajasarnaste vektorite hulga tähega $\tau$ ja defineerime hulgas $\tau$ seose $\sim$ järgnevalt. Kui $u,v \in \tau$, siis $u \sim v$ parajasti siis, kui $g\left(u,v\right) < 0$. Sedasi defineeritud seos $\sim$ on ekvivalents:
\begin{itemize}
\item[(a)] \textbf{refleksiivsus} järeldub vahetult ajasarnase vektori definitsioonist;
\item[(b)] seose $\sim$ \textbf{sümmeetrilisus} tuleneb skalaarkorrutise sümmeetrilisusest;
\item[(c)] \textbf{refleksiivsuseks} märgime, et kui $g\left(u,v\right) < 0$ ja $g\left(v, w\right) < 0$, siis teoreemi \ref{teoreem:ajasarnased_vektorid} põhjal $u^4 v^4 > 0$ ja $v^4 w^4 > 0$ ehk $u^4$ ja $v^4$ ning $v^4$ ja $w^4$ on sama märgiga ja seega $\sign u^4 = \sign w^4$, millest saame $u^4 w^4 > 0$. Rakendades nüüd veelkord teoreemi \ref{teoreem:ajasarnased_vektorid}, siis saame, et $g\left(u, w\right) < 0$.
\end{itemize}
Paneme tähele, et seos $\sim$ jagab hulga $\tau$ täpselt kaheks ekvivalentsiklassiks. Tõe\-poolest,
kui $u, v \in \M$ on ajasarnased, siis on meil teoreemi \ref{teoreem:ajasarnased_vektorid} põhjal kaks varianti. Peab kehtima kas 
\begin{align}
g\left(u,v\right) &< 0 \text{ või} \tag{1} \\
g\left(u,v\right) &> 0. \tag{2}
\end{align}
Kui kehtib võrratus (1), siis on $u \sim v$ ja korras. Vastupidi, kui $u$ ja $v$ jaoks kehtib (2), siis $u \nsim v$ ja piisab näidata, et kui $w \in \tau$ korral $g\left(u,w\right) > 0$, siis $v \sim w$. Võrratuse (2) kehtivuseks peavad $u, v$ ja $w$ teoreemi \ref{teoreem:ajasarnased_vektorid} põhjal rahuldama võrratusi $u^4 v^4 < 0$ ja $u^4 w^4 < 0$, mis tähendab, et arvud $u^4, v^4$ ja $u^4, w^4$ on erinevate märkidega. Seega $v^4$ ja $w^4$ on samade märkidega ja järelikult $g\left(v,w\right) < 0$ ehk $v \sim w$. \\
Neid kahte ekvivalentsiklassi tähistatakse $\tau^+$ ja $\tau^-$. Märgime, et elementide ekvivalentsiklassidesse $\tau^+$ ja $\tau^-$ jaotamine toimub meie postuleerimise täpsusega ja on meelevaldne.

\begin{teoreem}\textnormal{\cite[teoreem 1.3.3]{Naber}}
Olgu $\Lambda = \left[\Lambda\indices{^a_b}\right]_{a,b = 1,2,3,4}$ Lorentzi teisendus, see tähendab $\Lambda \in \L_{GH}$ ja olgu $\{e_a\}$ ruumi $\M$ orto\-normeeritud baas. Siis on järgmised väited samaväärsed:
\begin{itemize}
\item[(i)] $\Lambda$ on ortokroonne;
\item[(ii)] $\Lambda$ säilitab kõikide nullpikkusega vektorite ajakoordinaadi märgi, see tähendab, kui $u = u^a e_a$ on nullpikkusega, siis arvud $u^4$ ja $\hat{u}^4 = \Lambda\indices{^4_b} u^b$ on sama märgiga;
\item[(iii)] $\Lambda$ säilitab kõikide ajasarnaste vektorite ajakoordinaadi märgi.
\end{itemize}
\end{teoreem}

\begin{proof}
Olgu $u = u^a e_a \in \M$ ajasarnane või nullpikkusega vektor. Cauchy-Schwartz-Bunjakowski võrratusest ruumis $\R^{3}$ saame
\begin{align} \label{eq:ortokronkrit1}
\left(\Lambda\indices{^4_1} u^1 + \Lambda\indices{^4_2} u^2 + \Lambda\indices{^4_3} u^3\right)^2 \leq \left(\sum_{i = 1}^{3} \left(\Lambda\indices{^4_i}\right)^2 \right) \left(\sum_{j = 1}^{3} \left(u^j\right)^2 \right).
\end{align}
Sooritades nüüd võrduses \ref{eq:lambdaeta2} asenduse $a = b = 4$ saame
\begin{align} \label{eq:ortokronkrit2}
\left(\Lambda\indices{^4_1}\right)^2 + \left(\Lambda\indices{^4_2}\right)^2 &+ \left(\Lambda\indices{^4_3}\right)^2 - \left(\Lambda\indices{^4_4}\right)^2 = -1 \text{, milllest järleldub} \nonumber \\
\left(\Lambda\indices{^4_4}\right)^2 &> \left(\Lambda\indices{^4_1}\right)^2 + \left(\Lambda\indices{^4_2}\right)^2 + \left(\Lambda\indices{^4_3}\right)^2.
\end{align}
Et $u \neq 0$, siis tingimustest \ref{eq:ortokronkrit1} ja \ref{eq:ortokronkrit2} saame $\left(\Lambda\indices{^4_1} u^1 + \Lambda\indices{^4_2} u^2 + \Lambda\indices{^4_3} u^3\right)^2 < \left(\Lambda\indices{^4_4} u^4\right)^2$, mille võime ruutude vahe valemit kasutades kirjutada kujul
\begin{align} \label{eq:ortokronkrit3}
\left(\Lambda\indices{^4_1} u^1 + \Lambda\indices{^4_2} u^2 + \Lambda\indices{^4_3} u^3 - \Lambda\indices{^4_4} u^4\right)\left(\Lambda\indices{^4_1} u^1 + \Lambda\indices{^4_2} u^2 + \Lambda\indices{^4_3} u^3 + \Lambda\indices{^4_4} u^4\right) < 0.
\end{align}
Defineerides $v \in \M$ võrdusega $v = \Lambda\indices{^4_1} e^1 + \Lambda\indices{^4_2} e^2 + \Lambda\indices{^4_3} e^3 - \Lambda\indices{^4_4} e^4$ on $v$ ajasarnane vektor, kusjuures seose \ref{eq:ortokronkrit3} saame esitada kujul
\begin{align} \label{eq:ortokronkrit4}
g\left(u , v\right)\hat{u}^4 < 0.
\end{align}
Viimane võrratus ütleb meile, et arvudel $g\left(u,v\right)$ ja $\hat{u}^4$ on erinevad märgid.
Näitame viimaks, et $\Lambda\indices{^4_4} \geq 1$ siis ja ainult siis, kui arvudel $u^4$ ja $\hat{u}^4$ on samad märgid.
Selleks oletame esmalt, et $\Lambda\indices{^4_4} \geq 1$. Kui $u^4 > 0$, siis teoreemi \ref{teoreem:ajasarnased_vektorid} järgi $g\left(u, v\right) < 0$ ja seega \ref{eq:ortokronkrit4} põhjal $\hat{u}^4 >0$. Kui $u^4 < 0$, siis $g\left(u, v\right) > 0$, ja seega $\hat{u}^4 < 0$. Kokkuvõttes järeldub võrratusest $\Lambda\indices{^4_4} \geq 1$, et $u^4$ ja $\hat{u}^4$ on samamärgilised. Analoogiliselt saab näidata, et kui $\Lambda\indices{^4_4} \leq -1$, siis $u^4$ ja $\hat{u}^4$ on erimärgilised.
\end{proof}

Teoreemi tõestusest saame teha järgmise olulise järelduse.
\begin{jareldus}
Kui $\Lambda$ on mitteortokroonne, siis ta muudab kõikide ajasarnaste ja nullpikkusega vektorite ajakoordinaadi märgi.
\end{jareldus}

Järelduse tulemust arvestades on mõistlik edasises uurida vaid hulga $\L_{GH}$ elemente, mis on ortokroonsed. Kuna sellised Lorentzi teisendused ei muuda ajasarnaste ega nullpikkusega vektorite ajakoordinaadi märki, siis võime konkreetsuse mõttes piirduda vaid selliste ortonormaalsete baasidega $\{e_1, e_2, e_3, e_4\}$ ning $\{\hat{e}_1, \hat{e}_2, \hat{e}_3, \hat{e}_4\}$, kus $e_4 \sim \hat{e}_4$. Tuletame meelde, et siinjuures ei ole vahet kumba faktorhulga $\M / {\sim}$ ekvivalentsiklassi elemendid $e_4$ ja $\hat{e}_4$ kuuluvad, sest nende ekvivalentsiklasside täpne fikseerimine on meelevaldne.

Meenutades veel seost \ref{eq:lambdaetamat1}, siis saame kirjutada $\det\left(\Lambda^T \eta \Lambda\right) = \det \eta$, millest tuleneb $\left(\det \Lambda\right)^2 = 1$. Järelikult kehtib kas
\begin{equation*}
\det \Lambda = 1 \text{ või } \det \Lambda = -1,
\end{equation*}
ja seejuures ütleme, et $\Lambda$ on \emph{Lorentzi päristeisendus}, kui $\det \Lambda = 1$.

\begin{lause} \label{lause:ryhmL}
Hulk $\L := \{\Lambda \in \L_{GH} : \det \Lambda = 1,\ \Lambda\indices{^4_4} \geq 1\}$ on korrutamise suhtes rühma $\L_{GH}$ alamrühm, see tähendab ortokroonsed Lorentzi päristeisendusused moodustavad rühma.
\end{lause}
\begin{proof}
Veendumaks, et hulk $\L$ on rühm näitame, et ta on kinnine kor\-rutamise ja pöördelemendi võtmise suhtes. Olgu $\Lambda, \Lambda_1, \Lambda_2 \in \L$.
Paneme tähele, et $\det \left(\Lambda_1 \Lambda_2 \right) = \det \Lambda_1 \cdot \det \Lambda_2 = 1$ ja seose \ref{eq:lambda-1lambda-trans} põhjal $\det \Lambda^{-1} = \det \left(\eta \Lambda^T \eta\right) = (-1)^2 \det \Lambda^T = \det \Lambda = 1$. \\
Seose \ref{eq:lambda-1lambda-trans} põhjal on ilmne ka $\left(\Lambda^{-1}\right)\indices{^4_4} = \left(\Lambda^T\right)\indices{^4_4} = \Lambda\indices{^4_4} \geq 1$.
Veendumaks, et $\left(\Lambda_1 \Lambda_2\right)\indices{^4_4} \geq 1$ tähistame $\vec{a} = \left(\left(\Lambda_1\right)\indices{^4_1}, \left(\Lambda_1\right)\indices{^4_2}, \left(\Lambda_1\right)\indices{^4_2}\right)$ 
ja $\vec{b} = \left(\left(\Lambda_2\right)\indices{^1_4}, \left(\Lambda_2\right)\indices{^2_4}, \left(\Lambda_2\right)\indices{^3_4}\right)$ ning skalaarkorrustise $\vec{a} \cdot \vec{b}$ all mõtleme eukleidilist skalaarkorrutist ruumis $\R^3$. Selliseid tähistusi arvestades võime kirjutada
\begin{align*}
\left(\Lambda_1 \Lambda_2\right)\indices{^4_4} &= \left(\Lambda_1\right)\indices{^4_1}\left(\Lambda_2\right)\indices{^1_4} +  \left(\Lambda_1\right)\indices{^4_2}\left(\Lambda_2\right)\indices{^2_4} + \left(\Lambda_1\right)\indices{^4_3}\left(\Lambda_2\right)\indices{^3_4} + \left(\Lambda_1\right)\indices{^4_4}\left(\Lambda_2\right)\indices{^4_4} = \\
&= \vec{a} \cdot \vec{b} + \left(\Lambda_1\right)\indices{^4_4}\left(\Lambda_2\right)\indices{^4_4}.
\end{align*}
Võttes valemis \ref{eq:lambdaeta1} $a = b = 4$ saame $\left(\left(\Lambda_1\right)\indices{^4_4}\right)^2 - \vec{a}\cdot\vec{a} = 1$, millest 
\[|\vec{a}| = \sqrt{\left(\left(\Lambda_1\right)\indices{^4_4}\right)^2 - 1} \leq \left(\Lambda_1\right)\indices{^4_4}.\]
Analoogiliselt saame valemist \ref{eq:lambdaeta2}, et $|\vec{b}| \leq \left(\Lambda_2\right)\indices{^4_4}$ ja seega kokkuvõttes $|\vec{a}\cdot\vec{b}| \leq |\vec{a}||\vec{b}| \leq \left(\Lambda_1\right)\indices{^4_4} \left(\Lambda_2\right)\indices{^4_4}$. Viimane aga tähendab, et $\left(\Lambda_1 \Lambda_2\right) \geq 0$, millest piisab, et kehtiks $\left(\Lambda_1 \Lambda_2\right) \geq 1$. Tõepoolest, kui $\hat{\Lambda} \in \L_{GH}$ ja tähistame 
$\vec{c} = \left(\frac{\hat{\Lambda}\indices{^4_1}}{\hat{\Lambda}\indices{^4_4}}, \frac{\hat{\Lambda}\indices{^4_2}}{\hat{\Lambda}\indices{^4_4}}, \frac{\hat{\Lambda}\indices{^4_3}}{\hat{\Lambda}\indices{^4_4}}\right)$, 
siis \ref{eq:lambdaeta1} põhjal
\[ -1 = \left(\hat{\Lambda}\indices{^4_1}\right)^2 + \left(\hat{\Lambda}\indices{^4_2}\right)^2 + \left(\hat{\Lambda}\indices{^4_3}\right)^2 - \left(\hat{\Lambda}\indices{^4_4}\right)^2 = - \left(\hat{\Lambda}\indices{^4_4}\right)^2 \left(1 - \vec{c} \cdot \vec{c}\right). \]
Viimane võrdus saab kehtida vaid juhul, kui $1 - \vec{c} \cdot \vec{c} > 0$. Järelikult
\[\left(\left(\hat{\Lambda}\right)\indices{^4_4}\right)^2 = \frac{1}{1 - \vec{c}\cdot\vec{c}} \geq 1\]
ehk kehtib kas $\left(\hat{\Lambda}\right)\indices{^4_4} \geq 1$ või $\left(\hat{\Lambda}\right)\indices{^4_4} \leq 1$. Sellega oleme näidanud, et $\L$ on rühm.
\end{proof}
\begin{markus}
Rühma $\L$ lausest \ref{lause:ryhmL} nimetatakse sageli \emph{Lorentzi rühmaks}, nagu ka rühma $\L_{GH}$.
\end{markus}
Vaatleme järgnevalt lähemalt hulga $\L$ alamhulka $\mathcal{R}$, mille elemendid avalduvad kujul
\[R = \begin{pmatrix}
R\indices{^1_1} & R\indices{^1_2} & R\indices{^1_3} & 0 \\ 
R\indices{^2_1} & R\indices{^2_2} & R\indices{^2_3} & 0 \\  
R\indices{^3_1} & R\indices{^3_2} & R\indices{^3_3} & 0 \\ 
0 & 0 & 0 & 1 \\ 
\end{pmatrix},\]
kus $\left(R\indices{^i_j}\right)_{i,j = 1,2,3}$ on ortogonaalne ja unimodulaarne, see tähendab $\left(R\indices{^i_j}\right)^{-1} = \left(R\indices{^i_j}\right)^T$ ja $det \left(R\indices{^i_j}\right) = 1$. Märgime, et tõepoolest $R \in \L$, sest $R\indices{^4_4} = 1$ definitsiooni järgi ning kasutades Laplace'i teoreemi saame $\det R = 1 \cdot (-1)^{4+4} \cdot \det \left(R\indices{^i_j}\right) = 1$. Samuti on täidetud ortogonaalsuse tingimus \ref{eq:lambdaetamat1} kuna
\begin{align*}
R^T \eta R = 
\begin{pmatrix}
\left(R\indices{^i_j}\right) & 0 \\ 
0 & 1 \\ 
\end{pmatrix}^T \eta \begin{pmatrix}
\left(R\indices{^i_j}\right) & 0 \\ 
0 & 1 \\ 
\end{pmatrix} = \begin{pmatrix}
\left(R\indices{^i_j}\right)^T & 0 \\ 
0 & 1 \\ 
\end{pmatrix} \begin{pmatrix}
\left(R\indices{^i_j}\right) & 0 \\ 
0 & -1 \\ 
\end{pmatrix} = \\
= \begin{pmatrix}
\left(R\indices{^i_j}\right)^T \left(R\indices{^i_j}\right) & 0 \cdot 0 \\ 
0 \cdot 0  & -1 \cdot 1 \\ 
\end{pmatrix} = 
\begin{pmatrix}
\left(R\indices{^i_j}\right)^{-1} \left(R\indices{^i_j}\right) & 0 \\ 
0  & -1 \\ 
\end{pmatrix} = \begin{pmatrix}
E & 0 \\ 
0  & -1 \\ 
\end{pmatrix} = \eta.
\end{align*}
Paneme veel tähele, et hulk $\mathcal{R}$ on rühm. Tõepoolest, kui $R_1, R_2 \in \mathcal{R}$, siis 
\begin{align*}
R_1 R_2 = \begin{pmatrix}
\left(\left(R_1\right)\indices{^i_j}\right) & 0 \\ 
0 & 1 \\ \end{pmatrix} \begin{pmatrix}
\left(\left(R_2\right)\indices{^i_j}\right) & 0 \\ 
0 & 1 \\ \end{pmatrix} = 
\begin{pmatrix}
\left(\left(R_1\right)\indices{^i_j}\right) \left(\left(R_2\right)\indices{^i_j}\right) & 0 \\ 
0 & 1 \\ \end{pmatrix}
\end{align*}
ja seega $(R_1 R_2)\indices{^4_4} = 1$ ning $\det \left(\left(\left(R_1\right)\indices{^i_j}\right) \left(\left(R_2\right)\indices{^i_j}\right) \right) = \det \left(\left(R_1\right)\indices{^i_j}\right) \det \left(\left(R_1\right)\indices{^i_j}\right) = 1$.
Teisalt, kui $R \in \mathcal{R}$, siis valemist \ref{eq:lambda-1lambda-trans} saame
\begin{align*}
R^{-1} = \eta R^T \eta = \eta \begin{pmatrix}
\left(R\indices{^i_j}\right) & 0 \\ 
0 & 1 \\ \end{pmatrix}^T \eta = 
\eta \begin{pmatrix}
\left(R\indices{^i_j}\right)^T & 0 \\ 
0 & -1 \\ \end{pmatrix} = \begin{pmatrix}
\left(R\indices{^i_j}\right)^T & 0 \\ 
0 & 1 \\ \end{pmatrix}.
\end{align*}
Järelikult $\left(R^{-1}\right)\indices{^4_4} = 1$ ning $\det \left(\left(R^{-1}\right)\indices{^i_j}\right) = \det \left(R\indices{^i_j}\right)^T = \det \left(R\indices{^i_j}\right) = 1$. \\
Et hulga $\mathcal{R}$ elemendid kirjeldavad koordinaatide teisendusi, mis pööravad ruumikoordinaate, siis ütleme teisenduse $R \in \mathcal{R}$ kohta \emph{pööre} ja rühma $\mathcal{R}$ nimetame rühma $\L$ \emph{pöörete alamrühmaks}.

\begin{lause}
Olgu $\Lambda \in \L$. Siis on järgmised väited samaväärsed:
\begin{itemize}
\item[(i)] $\Lambda$ on pööre, see tähendab $\Lambda \in \mathcal{R}$;
\item[(ii)] $\Lambda\indices{^1_4} = \Lambda\indices{^2_4} = \Lambda\indices{^3_4} = 0$;
\item[(iii)] $\Lambda\indices{^4_1} = \Lambda\indices{^4_2} = \Lambda\indices{^4_3} = 0$;
\item[(iv)] $\Lambda\indices{^4_4} = 1$.
\end{itemize}
\end{lause}

\begin{proof}
Implikatsioonid $(i) \Longrightarrow (ii), (i) \Longrightarrow (iii)$ ja $(i) \Longrightarrow (ii)$ on pöörde definitsiooni arvestades ilmsed. \\
Näitamaks, et $(ii), (iii)$ ja $(iv)$ on samaväärsed märgime, et fikseerides seostes \ref{eq:lambdaeta1} ja \ref{eq:lambdaeta2} sobivalt indeksid saame
\begin{align*}
\left(\Lambda\indices{^1_4}\right)^2 + \left(\Lambda\indices{^2_4}\right)^2 + \left(\Lambda\indices{^3_4}\right)^2 - \left(\Lambda\indices{^4_4}\right)^2 = -1, \\
\left(\Lambda\indices{^4_1}\right)^2 + \left(\Lambda\indices{^4_2}\right)^2 + \left(\Lambda\indices{^4_3}\right)^2 - \left(\Lambda\indices{^4_4}\right)^2 = -1.
\end{align*}
Võttes arvesse, et $\Lambda$ on ortokroonne, siis $\Lambda\indices{^4_4} \geq 1$ ja seega saame kirjutada
\begin{align*}
\left(\Lambda\indices{^1_4}\right)^2 + \left(\Lambda\indices{^2_4}\right)^2 + \left(\Lambda\indices{^3_4}\right)^2 - 1 \geq -1, \\
\left(\Lambda\indices{^4_1}\right)^2 + \left(\Lambda\indices{^4_2}\right)^2 + \left(\Lambda\indices{^4_3}\right)^2 - 1 \geq -1,
\end{align*}
mis saab aga võimalik olla vaid juhul, kui $\Lambda\indices{^1_4} = \Lambda\indices{^2_4} = \Lambda\indices{^3_4} = \Lambda\indices{^4_1} = \Lambda\indices{^4_2} = \Lambda\indices{^4_3} = 0$.

Tõestuse lõpetamiseks jääb veel näidata, et kui $\Lambda \in \L$ ja kehtib $(ii)$ (ja seega ka $(ii)$ ning $(iv)$), siis $\left(\Lambda\indices{^i_j}\right)_{i,j = 1, 2, 3}$ on ortogonaalne ja unimodulaarne. Võrdus $\det \left(\Lambda\indices{^i_j}\right) = 1$ on tõestatud osas kus näitasime, et $\mathcal{R}$ on rühm ja seega on $\left(\Lambda\indices{^i_j}\right)$ ortogonaalne. Jääb veel näidata, et $\left(\Lambda\indices{^i_j}\right)^{-1} = \left(\Lambda\indices{^i_j}\right)^T$. Selleks märgmine, et blokkmaatriksi pööramise eeskirja kohaselt
\begin{align*}
\Lambda^{-1} = 
\begin{pmatrix}
\left(\Lambda\indices{^i_j}\right) & 0 \\
0 & 1 \\ 
\end{pmatrix}^{-1} = \qquad\qquad\qquad\qquad\qquad\qquad\qquad\quad\\
\begin{pmatrix}
\left(\Lambda\indices{^i_j}\right)^{-1} + \left(\Lambda\indices{^i_j}\right)^{-1} 0  \left(1-0 \left(\Lambda\indices{^i_j}\right)
^{-1} 0 \right)^{-1} 0 \left(\Lambda\indices{^i_j}\right)^{-1} & -\left(\Lambda\indices{^i_j}\right)^{-1} 0 \left(1 - 0 \left
(\Lambda\indices{^i_j}\right)^{-1} 0 \right)^{-1} \\
-\left(1 - 0 \left(\Lambda\indices{^i_j}\right)^{-1} 0\right)^{-1} 0 \left(\Lambda\indices{^i_j}\right)^{-1} & \left(1 - 0 \left
(\Lambda\indices{^i_j}\right)^{-1} 0 \right)^{-1} \\
\end{pmatrix} \\
 = \begin{pmatrix}
\left(\Lambda\indices{^i_j}\right)^{-1} & 0 \\
0 & 1^{-1} \\ 
\end{pmatrix} = 
\begin{pmatrix}
\left(\Lambda\indices{^i_j}\right)^{-1} & 0 \\
0 & 1 \\ 
\end{pmatrix}. \qquad\qquad\qquad\qquad\qquad\qquad
\end{align*}
Osas kus näitasime, et $\mathcal{R}$ on rühm tuli ka välja, et $\Lambda^{-1} = \begin{pmatrix} \left(\Lambda\indices{^i_j}\right)^T & 0 \\ 0 & 1 \\ \end{pmatrix}$, millest $\left(\Lambda\indices{^i_j}\right)^{-1} = \left(\Lambda\indices{^i_j}\right)^T$ ehk $\left(\Lambda\indices{^i_j}\right)$ on ortogonaalne. Kokkuvõttes oleme näidanud, et $\Lambda$ on pööre, nagu tarvis.
\end{proof}

\subsection{Poincar\'e rühm}

Arvestades Pythagorase teoreemi kahemõõtmelisel tasandil toome sisse sündmuste kauguse mõiste üldistatuna Minkowski aegruumile.

\begin{definitsioon}
Olgu $x = x^a$ ja $y = y^a$, $a = 1, 2, 3, 4$, kaks sündmust Minkowski ruumis $\M$. Sündmuste $x$ ja $y$ vaheliseks \emph{kauguseks} nimetame reaalarvu $s$, mis on defineeritud valemiga
\begin{align*}
s = s\left(x, y\right) = \eta_{ab} \left(x^a - y^a\right)\left(x^b - y^b\right), a,b = 1, 2, 3, 4,
\end{align*}
kus $\eta$ on Minkowski ruumi meetrika.
\end{definitsioon}

Meenutame, et Minkowski ruumi pseudoortogonaalteisendus $L$ on defineeritud kui kujutus, mis säilitab skalaarkorrutise $g$. Olgu $x, y \in \M$. Osutub, et kujutuse $L$ ortogonaalsusest järeldub lihtsasti, et sündmuste $x$ ja $y$ vaheline kaugus sama, mis sündmustel $Lx$ ja $Ly$, see tähendab $s\left(x, y\right) = s\left(Lx, Ly\right)$. Tõepoolest, kui $\Lambda$ on kujutuse $L$ maatriks, siis valemi \ref{eq:lambdaeta1} abil saame kirjutada
\begin{align*}
s\left(Lx, Ly\right) &= \eta_{ab} \left(\Lambda\indices{^a_c} x^c - \Lambda\indices{^a_c} y^a\right) \left(\Lambda\indices{^b_d} x^d - \Lambda\indices{^b_d} y^d\right) = \\
&= \eta_{ab} \Lambda\indices{^a_c} \left(x^c - y^c\right) \Lambda\indices{^b_d} \left(x^d - y^d\right) = \\
&= \eta \Lambda\indices{^a_c} \Lambda\indices{^b_d} \left(x^c - y^c\right) \left(x^d - y^d\right) = \\
&= \eta \left(x^c - y^c\right) \left(x^d - y^d\right) = s(x, y).
\end{align*}

Olgu $n \in \Mat_{4, 1} \R$ komponentidega $n^a, a = 1, 2, 3, 4$ mingi fikseeritud $1 \times 4$ reaalne maatriks. On väga loomulik oletada, et sündmuste $x$ ja $y$ jäigal nihutamisel
\begin{align} \label{eq:niked}
\hat{x}^b = x^b + n^b, \quad \hat{y}^b = y^b + n^b, \quad b = 1, 2, 3, 4,
\end{align}
ruumis $\M$ nende sündmuste vaheline kaugus ei muutu. Arvestades meie antud kauguse definitsiooni see nii ilmselt ka on. Seejuures teisendusi \ref{eq:niked} nimetame \emph{nihketeisendusteks}.

Kokkuvõttes oleme leidnud kaks kujutuste klassi, nihked ja (homogeensed) Lorentzi teisendused, mille suhtes ruumi $\M$ sündmuste vaheline kaugus on invariantne. Üldisemalt nimetame teisendust, mille suhtes sündmuste vaheline kaugus on invariantne suurus \emph{sümmeetriateisenduseks}. Pöörame järgnevas pisut tähelepanu sellistele sümmeetria\-teisendustele, mille me saame kui vaatleme nihkeid koos Lorentzi teisendustega.

\begin{definitsioon}
Olgu $\Lambda \in \L_{GH}$ Lorentzi teisendus ja $n \in \Mat_{4, 1}\R$ mingi nihketeisenduse maatriks. \emph{Poincar\'e teisenduseks\footnote{Vahel kasutatakse \emph{Poicar\'e teisenduse} asemel terminit \emph{Mittehomogeenne Lorentzi teisendus}}} nimetame kujutust kujul
\begin{align*}
\M \ni x \mapsto \Lambda\indices{^a_b} x^b + n^b \in \M
\end{align*}
ja tähistame paarina $\{n, \Lambda\}$.
\end{definitsioon}

\begin{definitsioon} \textnormal{\cite{Barut}}
Kõikide Poincar\'e teisenduste hulka
\begin{align*}
P = \left\lbrace \{n, \Lambda\} : n \in \Mat_{4, 1}\R, \Lambda \in \L_{GH} \right\rbrace
\end{align*}
koos tehtega $\circ : P \times P \rightarrow P,\ \{n_1, \Lambda_1\} \circ \{n_2, \Lambda_2\} \mapsto \{n_1 + \Lambda_1 n_2, \Lambda_1 \Lambda_2\}$, nimetame \emph{Poincar\'e rühmaks}. Seda rühma tähistatakse sümboliga $\P$.
\end{definitsioon}

Veendumaks Poincar\'e rühma defintsiooni korrektsuses piisab näidata, et hulk $P$ varustatuna tehtega $\circ : P \times P \rightarrow P$ moodustab tõepoolest rühma.
Selle tõestamiseks olgu meil fikseeritud kolm Poicar\'e teisendust $\{n_1, \Lambda_1\}, \{n_2, \Lambda_2\}$ ja $\{n_3, \Lambda_3\}$. Esiteks märgime, et $ \Lambda_1 n_2 \in \Mat_{4, 1}\R$ ja seega ka $n_1 + \Lambda_1 n_2 \in \Mat_{4, 1}\R$ ja $\Lambda_1 \Lambda_2 \in \L_{GH}$, sest $\L_{GH}$ on rühm maatriksite korrutamise suhtes ja seega $\{n_1, \Lambda_1\} \circ \{n_2, \Lambda_2\} \in P$.
Assotsiatiivsus on vahetult kontrollitav:
\begin{align*}
\left(\{n_1, \Lambda_1\} \circ \{n_2, \Lambda_2\}\right) \circ \{n_3, \Lambda_3\} &= \{n_1 + \Lambda_1 n_2, \Lambda_1 \Lambda_2 \} \circ \{n_3, \Lambda_3 \} = \\
= \{n_1 + \Lambda_1 n_2 + \Lambda_1 \Lambda_2 n_3, \Lambda_1 \Lambda_2 \Lambda_3 \} &= \{n_1 + \Lambda_1 \left( n_2 + \Lambda_2 n_3 \right), \Lambda_1 \left( \Lambda_2 \Lambda_3 \right)\} = \\
= \{n_1, \Lambda_1 \} \circ \{n_2 + \Lambda_2 n_3, \Lambda_2 \Lambda_3 \} &= \{n_1, \Lambda_1\} \circ \left( \{n_2, \Lambda_2\} \circ \{n_3, \Lambda_3\} \right)
\end{align*}
Lihtne on näha, et ühikuks sobib võtta $\{0, E\}$ ja seega on Poincar\'e teisenduse $\{n_1, \Lambda_1\}$ pöördelemendiks teisendus $\{ -\left(\Lambda_1\right)^{-1} n_1, \left(\Lambda_1\right)^{-1} \}$, sest
\[ n_1 + \Lambda_1 \left(-\left(\Lambda_1\right)^{-1}\right) n_1 = n_1 - E n_1 = 0 \text{ ja } \Lambda_1 \left(\Lambda_1\right)^{-1} = E. \]
Kokkuvõttes oleme näidanud, et Poincar\'e rühm $\P$ on oma algebralistelt omadustelt tõepoolest rühm.

Kuna teisendusi, mis säilitavad punktide vahelisi kaugusi nimetatakse isomeetriateisendusteks ja Poincar\'e teisendused säilitavad ruumi $\M$ sündmuste vahel kaugused, siis öeldakse Poincar\'e rühma kohta \emph{Minkowski ruumi isomeetirarühm}.

\begin{markus}
Paneme tähele, et kui vaatleme selliseid Poincar\'e teisndusi $\{n_1, \Lambda_1\}$ ja $\{n_2, \Lambda_2\}$, mille korral $n_1 = n_2 = 0$, siis on meil tegelikult tegu Lorentzi teisendustega ja seejuures nende teisenduste korrutised on samad Lorentzi ja Poincar\'e rühma kontekstis.
\end{markus}

Poincar\'e teisendusi, millel on kuju
\begin{align*}
\hat{x}^b = x^b + \delta x^b, \delta x^b = - \lambda\indices{^b_a} x^a + N^b,
\end{align*}
kus $\Lambda\indices{^b_d} \approx \delta^d_d - \lambda\indices{^b_d}$ ja $n^b \approx1 + N^b$, nimetame lõpmata väikesteks ehk \emph{infinitesimaalseteks}. 
Tingimusest \ref{eq:lambdaeta1}

\newpage
\section{Lie rühm ja Lie algebra}

\subsection{Lie algebra}

\begin{definitsioon} \label{def:lie-algebra}
Vektorruumi $L$ nimetatakse \emph{Lie algebraks}, kui on määratud bilineaarvorm $\left[\phantom{u},\phantom{u}\right] : L \times L \rightarrow L$, mis on antikommuteeruv, see tähendab kõikide $u, v \in L$ korral $\left[u, v\right] = - \left[v, u\right]$ ja iga $u, v, w \in L$ puhul kehtib Jacobi samasus ehk $\left[u, \left[v, w\right]\right] + \left[v, \left[w, u\right]\right] + \left[w, \left[u, v\right]\right] = 0$.
\end{definitsioon}

Bilineaarvormi $\left[\phantom{u}, \phantom{u}\right] : L \times L \rightarrow L$ Lie algebra definitsioonist nimetatakse \emph{kommutaatoriks}.

Piltlikult võib öelda, et kommutaator mõõdab kui palju on Lia algebra elemendid mittekommuteeruvad.

\begin{definitsioon}
Olgu $L$ lõplikumõõtmeline Lie algebra ja $\{t_i\}_{i = 1}^{r}$ selle Lie algebra vektoruumi bass. Kui tähistame $\left[t_i, t_j\right] = c_{ij}^{k} t_{k}$, kus $i,j \in \{1, 2, \ldots, r\}$, siis arve $c_{ij}^{k}$ nimetame Lie algebra \emph{struktuurikonstantideks}.
\end{definitsioon}

\begin{markus}
Lie algebrate kontekstis nimetatakse baasivektoreid sageli ka algebra \emph{generaatoriteks}.
\end{markus}

Lie algebra struktuurikonstantide definitsiooni põhjal on selge, et struktuurikonstantide väärtus sõltub Lie algebra baasi valikust. Lisaks saame vahetult kommutaatori antikommuteeruvusest
\begin{align*}
c_{ij}^{k} t_k = \left[t_i, t_j\right] = - \left[t_j, t_i\right] = - c_{ji}^{k} t_k
\end{align*}
ehk
\begin{align} \label{eq:struktuurikonst-antikomm}
c_{ij}^{k} = -c_{ji}^{k},
\end{align}
kus $t_i, t_j, t_k$ on Lie algebra generaatorid.

Kasutades Jacobi samasust baasivektoritele tuletame nüüd seosed struktuurikonstantide jaoks: 
\begin{align*}
\left[t_i, \left[t_j, t_k\right]\right] + \left[t_j, \left[t_k, t_i\right]\right] + \left[t_k, \left[t_i, t_j\right]\right] = 0, \\
\left[t_i, c_{jk}^{l}t_l\right] + \left[t_j, c_{ki}^{l}t_l\right] + \left[t_k, c_{ij}^{l}t_l\right] = 0, \\
c_{jk}^{l} \left[t_i, t_l\right] + c_{ki}^{l} \left[t_j, t_l\right] + c_{ij}^{l} \left[t_k, t_l\right] = 0, \\
c_{jk}^{l} c_{il}^{m} t_m + c_{ki}^{l} c_{jl}^{m} t_m + c_{ij}^{l} c_{kl}^{m} t_m = 0, \\
c_{jk}^{l} c_{il}^{m} + c_{ki}^{l} c_{jl}^{m} + c_{ij}^{l} c_{kl}^{m} = 0.
\end{align*}

Kokkuvõttes saame Lie algebra generaatorite kohta sõnastada järgmise tähtsa tulemuse.
\begin{lause}
Lie algebra on täielikult määratud tema generaatorite kommutatsiooni\-eeskirjadega ehk struktuurikonstantidega.
\end{lause}

\begin{naide} \label{naide:su2}
Vaatleme teist järku ruutmaatriksite hulka $\su2 \subset \Mat_2 \C$, kus $u \in \su2$ rahuldab tingimusi
\begin{align}
\Tr u &= 0, \label{eq:su2-tr=0}\\ 
u^\dag + u &= 0. \label{eq:su2-antihermitian}
\end{align}
Siin $u^\dag = \bar{u}^T$ ja maatirksit $u$, mille korral tingimus \ref{eq:su2-antihermitian} kehtib, nimetatakse \emph{anti-Hermite'i maatriksiks}.
 
Veenudme, et $\su2$ koos tavalise maatrikskommutaatoriga $\left[u,v\right] = uv - vu$, $u, v \in \su2$, on Lie algebra.

Olgu $u, v, w \in \su2$. Kõigepealt märgime, et 
\begin{align*}
\left(u + v\right) + \left(u + v\right)^\dag &= u + v + u^t + v^t = 0, \\
\Tr \left(u + v\right) &= 0
\end{align*}
ehk $u + v \in \su2$. Veendumaks, et $\su2$ on Lie algebra piisab nüüd veel näidata, et kommutaator $\left[\phantom{u}, \phantom{u}\right]$ rahuldab definitsioonis \ref{def:lie-algebra} antud tingimusi ning $\left[u,v\right] \in \su2$ ja $\Tr \left[u,v\right] = 0$. 

Paneme tähele, et definitsiooni järgi $\left[u,v\right] = uv - vu = - \left(vu - uv\right) = - \left[v,u\right]$ ning 
kehtib Jacobi samasus, sest
\begin{align*}
&\left[u, \left[v, w\right]\right] + \left[v, \left[w, u\right]\right] + \left[w, \left[u, v\right]\right] = \\
&= \left[u, vw - wv \right] + \left[v, wu - uw \right] + \left[w, uv - vu \right] = \\
&= \left[u, vw\right] - \left[u, wv\right] + \left[v, wu\right] - \left[u, uw\right] + \left[w, uv\right] - \left[w, vu\right] = \\
&= uvw - vwu - uwv + wvu + vwu - wuv - \\
&- vuw + uwv + wuv - uvw - wvu + vuw = 0.
\end{align*}
Seega on definitsiooni \ref{def:lie-algebra} eeldused täidetud. Teoreemi \ref{teor:trace} põhjal $\Tr \left(uv\right) = \Tr \left(vu\right)$, millest saame $\Tr \left[u,v\right] = 0$. Lõpuks paneme veel tähele, et $\left[u,v\right] + \left[u,v\right]^\dag = 0$. Tõepoolest, kuna $u^\dag = -u$ ja $v^\dag = -v$, siis
\begin{align*}
\left[u,v\right] + \left[u,v\right]^\dag &= uv - vu + \left(uv - vu\right)^\dag = uv - vu + \left(uv\right)^\dag - \left(vu\right)^\dag = \\
&= uv - vu + v^\dag u^\dag - u^\dag v^\dag = uv - vu + vu - uv = 0.
\end{align*}
\end{naide}

Näites \ref{naide:su2} toodud Lie algebrat $\su2$ nimetatakse \emph{teist järku spetsiaalsete unitaarsete maatriksite} Lie algebraks. Osutub, et Yang-Millsi väljateoorias\footnote{Yang-Millsi väljateooria on kalibratsiooniväljateooria, mille eesmärk on kirjeldada mittekommutatiivse rühmateooria, kvantkromodünaamika ja nõrga ning elektromagneetilise vastasmõju ühendamisel elementaarosakesi.} on Lie algebral $\su2$ väga tähtis roll seoses Lagrangian'i sümmeetriatega\footnote{Lagrangian ehk Lagrange'i funktsioon on funktsioon, mis kirjeldab süsteemi dünaamikat. Lagrangian on oma nime saanud itaalia matemaatiku Joseph-Louis Lagrange ($1736 - 1813$) järgi.}. Teiselt poolt, käesoleva töö seisukohast on vahest isegi olulisem märkida, et Lie algebrat $\su2$ kasutatakse Lorentzi rühma spiinoresituse konstrueerimiseks.

Uurime edasises veel pisut Lie algebra $\su2$ omadusi. Tingimusest \ref{eq:su2-tr=0} saame, et $u \in \su2$ peab omama kuju 
$u = \begin{pmatrix} a & b \\ c & -a\end{pmatrix}$. Arvestades ka tingimust \ref{eq:su2-antihermitian}, peab seega kehtima
\[\begin{pmatrix} a & b \\ c & -a\end{pmatrix} + \begin{pmatrix} \overline{a} & \overline{c} \\ \overline{b} & \overline{-a} \end{pmatrix} = 0,\]
millest saame järgmised tingimused:
\begin{align*}
\begin{cases}
    a + \overline{a} = 0, \\
    b + \overline{c} = 0. \\
  \end{cases}
\end{align*}
Võttes $a = \upsilon + i\varphi$, siis peab kehtima $\upsilon + i\varphi + \upsilon - i\varphi = 0$ ehk $\upsilon = 0$, millest $a = i\varphi$. Võttes veel arvesse, et $c = -\overline{b}$, siis saab maatriks $u$ kuju $u = \begin{pmatrix} i\varphi & b \\ -\overline{b} & -i\varphi \end{pmatrix}$, kus $b=x+iy \in \C$ ja $x, y, \varphi \in \R$. Viimane aga tähendab, et maatriksis $u$ lineaarselt sõltumatuid reaalseid komponente kolm ehk $\dim_{\R} \su2 = 3$. Arvestades tähistust $b = x + iy$ saame kirjutada
\begin{align*}
u = \varphi \begin{pmatrix} i & 0 \\ 0 & -i \end{pmatrix} + x \begin{pmatrix} 0 & 1 \\ -1 & 0 \end{pmatrix} + y \begin{pmatrix} 0 & i \\ i & 0 \end{pmatrix}.
\end{align*}
Sellega oleme leidnud Lie algebra $\su2$ baasi ehk generaatorid:
\begin{align*}
\rho_1 = \begin{pmatrix} 0 & i \\ i & 0 \end{pmatrix},\ \rho_2 = \begin{pmatrix} 0 & 1 \\ -1 & 0 \end{pmatrix},\ \rho_3 = \begin{pmatrix} i & 0 \\ 0 & -i \end{pmatrix}.
\end{align*}

Paneme tähele, et korrutades $\su2$ generaatoried $\rho_1, \rho_2, \rho_3$ arvuga $-i$ ning tähistame saadud maatriksid vastavalt $\sigma_1, \sigma_2, \sigma_3$, siis saame maatriksid
\begin{align*}
\sigma_1 = -i \cdot \rho_1 = \begin{pmatrix} 0 & 1 \\ 1 & 0 \end{pmatrix}, \ \sigma_2 = -i \cdot \rho_2 = \begin{pmatrix} 0 & -i \\ i & 0 \end{pmatrix},\ \sigma_3 = -i \cdot \rho_3 = \begin{pmatrix} 1 & 0 \\ 0 & 1 \end{pmatrix},
\end{align*}
mis on täpselt \emph{Pauli maatriksite}\footnote{Wolfgang Ernst Pauli ($1900 - 1958$) - austria füüsik, üks kvantmehaanika rajaid.} baas. Pauli maatriksitel on teoreetilises füüsikas väga tähtis roll.

\subsection{Lie algebra esitus}

\begin{definitsioon}
Olgu $V$ vektorruum ja $L$ Lie algebra. Lineaarkujutust $\psi : L \rightarrow \Lin V$ nimetatakse Lie algebra $L$ esituseks, kui ta säilitab kommutaatori, see tähendab suvaliste $u, v \in L$ korral $\left[\psi\left(u\right), \psi\left(v\right)\right] = \psi\left(\left[u, v\right]\right)$. 
\end{definitsioon}

\begin{definitsioon}
Lie algebra $L$ esitust $\psi : L \rightarrow \Mat_n \R$ nimetatakse \emph{maatriksesituseks}.
\end{definitsioon}

\begin{lause} \label{lause:adjoint}
Olgu $L$ Lie algebra. Kujutus $\ad : L \rightarrow \Lin L$, $\ad u \left(v\right) := \left[u, v\right]$, kus $u,v \in L$, on Lie algebra L esitus.
\end{lause}

\begin{proof}
Arvestades kommutaatori lineaarsust on ja seda kuidas me kujutuse $\ad$ defineerisime, on lineaarsus ilmne.

Veendumaks, et kujutus $\ad : L \rightarrow \Lin L$ on tõepoolest Lie algebra esitus, tuleb veel kontrollida, et kehtib võrdus $\left[\ad u_1, \ad u_2 \right] \left(v\right) = \ad \left[u_1, u_2 \right] \left(v\right)$. Selle võrduse saame annab meile Jacobi samasus, kuna
\begin{align*}
0 &= \left[ u_1, \left[u_2, v\right] \right] + \left[ u_2, \left[v, u_1 \right] \right] + \left[ v, \left[u_1, u_2\right] \right] = \\
&= \left[ u_1, \left[u_2, v\right] \right] + \left[ u_2, -\left[u_1, v \right] \right] - \left[\left[u_1, u_2\right], v\right] = \\
&= \left[ u_1, \left[u_2, v\right] \right] - \left[ u_2, \left[u_1, v \right] \right] - \left[\left[u_1, u_2\right], v\right],
\end{align*}
ehk $\left[ u_1, \left[u_2, v\right] \right] - \left[ u_2, \left[u_1, v \right] \right] = \left[\left[u_1, u_2\right], v\right]$ ja seega
\begin{align*}
\left[\ad u_1, \ad u_2 \right] \left(v\right) = \left[u_1, \left[u_2, v\right]\right ] - \left[u_2, \left[u_1, v\right]\right] = \left[\left[u_1, u_2\right], v\right] = 
\ad \left[u_1, u_2 \right] \left(v\right),
\end{align*}
ehk $\ad : L \rightarrow \Lin L$ on tõepoolest Lie algebra $L$ esitus.
\end{proof}

\begin{definitsioon}
Esitust lausest \ref{lause:adjoint} nimetame Lie algebra $L$ \emph{adjungeeritud esituseks}.
\end{definitsioon}

\begin{naide}
Leiame Lie algebra $\su2$ baasi $\{\rho_1, \rho_2, \rho_3\}$ adjungeeritud esituse
\[\{\ad \rho_1, \ad \rho_2, \ad \rho_3\}\]
maatrikskuju. Arvestades adjungeeritud esituse definitsiooni, tuleb meil selleks arvutada maatriksid $\ad \rho_1 \left(\rho_1\right), \ad \rho_1 \left(\rho_2\right), \ldots, \ad \rho_3 \left(\rho_2\right)$ ja $\ad \rho_3 \left(\rho_3\right)$. Teemegi seda:

\begin{align*}
\ad \rho_1 \left(\rho_1\right) &= \left[\rho_1, \rho_1\right] = \rho_1 \rho_1 - \rho_1 \rho_1 = 0, \\
\ad \rho_1 \left(\rho_2\right) &= \left[\rho_1, \rho_2\right] = \rho_1 \rho_2 - \rho_2 \rho_1 = -2\rho_3, \\
\ad \rho_1 \left(\rho_3\right) &= \left[\rho_1, \rho_3\right] = \rho_1 \rho_2 - \rho_3 \rho_1 = 2\rho_2, \\
\ad \rho_2 \left(\rho_1\right) &= \left[\rho_2, \rho_1\right] = - \left[\rho_1, \rho_2\right] = - \ad \rho_1 \left(\rho_2\right) = 2\rho_3, \\
\ad \rho_2 \left(\rho_2\right) &= \left[\rho_2, \rho_2\right] = \rho_2 \rho_2 - \rho_2 \rho_2 = 0, \\
\ad \rho_2 \left(\rho_3\right) &= \left[\rho_2, \rho_3\right] = \rho_2 \rho_2 - \rho_3 \rho_2 = -2\rho_1, \\
\ad \rho_3 \left(\rho_1\right) &= \left[\rho_3, \rho_1\right] = - \left[\rho_1, \rho_3\right] = - \ad \rho_1 \left(\rho_3\right) = -2\rho_2, \\
\ad \rho_3 \left(\rho_2\right) &= \left[\rho_3, \rho_2\right] = - \left[\rho_2, \rho_3\right] = - \ad \rho_2 \left(\rho_3\right) = 2\rho_1, \\
\ad \rho_3 \left(\rho_3\right) &= \left[\rho_3, \rho_3\right] = \rho_3 \rho_3 - \rho_3 \rho_3 = 0.
\end{align*}

Saadud võrduste põhjal võime nüüd kirjutada

\begin{align*}
\ad \rho_1 = 2\begin{pmatrix}
0 & 0 & 0 \\
0 & 0 & 1 \\
0 & -1 & 0
\end{pmatrix}, 
\ad \rho_2 = 2\begin{pmatrix}
0 & 0 & -1 \\
0 & 0 & 0 \\
1 & 0 & 0
\end{pmatrix}, 
\ad \rho_3 = 2\begin{pmatrix}
0 & 1 & 0 \\
-1 & 0 & 0 \\
0 & 0 & 0
\end{pmatrix},
\end{align*}

mis tähendab, Lie algebra $\su2$ adjungeeritud esituse baasiks sobib kolmandat järku antisümmeetriliste maatriksite süsteem 
\[\left\lbrace \begin{pmatrix}
0 & 0 & 0 \\
0 & 0 & 1 \\
0 & -1 & 0
\end{pmatrix}, \begin{pmatrix}
0 & 0 & -1 \\
0 & 0 & 0 \\
1 & 0 & 0
\end{pmatrix}, \begin{pmatrix}
0 & 1 & 0 \\
-1 & 0 & 0 \\
0 & 0 & 0
\end{pmatrix} \right\rbrace.\]
\end{naide}

Osutub, et saadud maatriksid moodustavad $3$-mõõtmelise ruumi pöörete rühma Lie algebra baasi. Sellega oleme sisuliselt näidanud, et tegelikult on Lie algebra $\su2$ ja pöörete rühma vahel väga tähtis seos.

\subsection{Lie rühm}

\begin{definitsioon}
Rühma $G$ nimetatakse \emph{Lie rühmaks}, kui ta on diferentseeruv muutkond ja tema (rühma) tehe on diferentseeruv.
\end{definitsioon}

\begin{definitsioon}
Olgu $G$ Lie rühm ja $e$ selle rühma ühikelement. Rühma $G$ puutujaruumi $T_e G$ punktis $e$ nimetatakse Lie rühma $G$ (poolt määratud) \emph{Lie algebraks}.
\end{definitsioon}

Osutub, et eespool vaatluse all olnud Lorentzi rühm $\L_{GH}$ ja Poincar\'e rühm $\P$ on oma matemaatilistelt omadustelt koguni Lie rühmad. Et selle väite formaalne tõestamine nõuab palju tehnilist tööd ja tulemusi diferentsiaalgeomeetriast ning Lie algebrate teooriast, siis jätame siinkohal range tõestuse andmata ning piirdume vaid fakultatiivse tulemusega.

Et edasist teooriaarendust oluliselt lihtsustada, on otstarbekas piirduda vaid Lie maatriksrühmade vaatlemisega. Samas märgime, et kuna maatriksrühmad moodustavad väga suure ja olulise rühmade klassi, siis kõik tähtsamad rühmad ja selles töös käsitletavad juhud jäävad kaetuks ka pärast sellise kitsenduse lisamist. Niisiis edasises eeldame, et kui tegelist on Lie rühmaga, siis see on maatriksrühm.

Järgnevalt uurime kuidas tekib Lie rühma poolt määratud Lie algebra kommutaator. Olgu $G$ Lie rühm ning $e$ selle rühma ühikelement. Kuna $G$ on Lie rühm, siis definitsiooni järgi on $G$ ka muutkond. Muutkonnal $G$ võime vaadelda parameetrilist joont $g\left(t\right) \in G$, $t \in \left(-\delta, \delta\right)$, kus $\delta \in \R$ on mingi fikseeritud arv ja nõuame, et $g\left(0\right) = e$. Sel viisil saame muutkonna $G$ puutujavektori $v = g'\left(0\right) \in T_e G$ punktis $e$.

Fikseerime elemendi $h \in G$ ning tekitame loomuliku viisil automorfismi: $g \mapsto h \cdot g \cdot h^{-1}$. Nüüd saame vaadelda joone $h \cdot g\left(t\right) h^{-1}$ puutujavektorit $v_h$. Meie eesmärk oleks välja selgitada millises vahekorras on puutujavektorid $v$ ja $v_h$.

Oletame järgnevalt, et meil on muutkonnal $G$ antud kaks parameetrilist joont $g\left(t\right)$ ja $h\left(t\right)$, $t \in \left(-\delta, \delta\right)$, selliselt, et $g\left(0\right) = h\left(0\right) = e$. Sel viisil võime vaadelda kommutanti $g\left(t\right) h\left(t\right) g^{-1}\left(t\right) h^{-1}\left(t\right)$, kusjuures on selge, et kommutatiivse rühma korral annab selline kommutant tulemuseks alalti ühikelemendi $e$. Lie rühmade teooriast on hästi teada\footnote{Vaata näiteks \cite[Exponential map]{Kirillov}} fakt, et kui $G$ on Lie algebra, siis eksisteerib sürjektsioon $\exp : T_e G \rightarrow G$, kus $T_e G$ on rühma $G$ ühikelemendi puutujatasandi Lie algebra. Seega saame leida $A, B \in T_e G$ selliselt, et $g\left(t\right) = e^{tA}$ ja $h\left(t\right) = e^{tB}$. 

Nõnda saame kommutandi $g\left(t\right) h\left(t\right) g^{-1}\left(t\right) h^{-1}\left(t\right)$ arvutamise taandada juba Lie algebrale $T_e G$, sest arvestades eelnevat ja maatrikseksponendi omadusi, kehtib võrdus $g\left(t\right) h\left(t\right) g^{-1}\left(t\right) h^{-1}\left(t\right) = e^{tA} e^{tB} e^{-tA} e^{-tB}$. Viimast oskame aga juba välja arvutada:
\begin{align*}
&\hphantom{=}\ \ g\left(t\right) h\left(t\right) g^{-1}\left(t\right) h^{-1}\left(t\right) = e^{tA} e^{tB} e^{-tA} e^{-tB} = \\
&= \left(E + At + \frac{1}{2}A^2 t^2 + \ldots \right) \left(E + Bt + \frac{1}{2}B^2 t^2 + \ldots \right) \\
&\hphantom{=}\ \left(E - At + \frac{1}{2}A^2 t^2 - \ldots \right) \left(E - Bt + \frac{1}{2}B^2 t^2 - \ldots \right) = \\
&= E + \left(A + B - A - B\right)t + \\
&+ \left(AB - A^2 - AB - BA - B^2 + AB + \frac{1}{2} A^2 + \frac{1}{2} B^2 + \frac{1}{2} A^2 + \frac{1}{2} B^2\right)t^2 \\
&+ \ldots = E + 0 + \left(AB - BA\right)t^2 + \ldots = E + \left[A, B\right]t^2 + \ldots
\end{align*}

Niisiis, teises lähenduses peab kehtima $g\left(t\right) h\left(t\right) g^{-1}\left(t\right) h^{-1}\left(t\right) = E + \left[A, B\right]t^2$.

\textcolor{red}{Põhjenda siin natuke kirjapandut!}

\subsection{Poincar\'e algebra kommutatsioonieeskirjad}


\newpage
\section*{Summary}
\addcontentsline{toc}{section}{Summary}
Siia tuleb ingliskeelne kokkuvõte...

\newpage
\addcontentsline{toc}{section}{Viited}
\bibliographystyle{alpha}
\bibliography{references}

\end{document}
